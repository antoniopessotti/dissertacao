%% ------------------------------------------------------------------------- %%
\chapter{Introdução} %Nome do capítulo.
\label{cap:intro} 

  \section{O som e o áudio}

  \section{Código aberto, software livre e cultura livre}

      \subsection{Cultura de compartilhamento}

      \subsection{Licenças livres para software e para outras mídias}
	  \subsubsection{Principais licenças}
	  \label{sec:princ_licencas}
    
	  \subsubsection{Domínio público e outras categorias}

      \subsection{A convergência das mídias e o multimídia}
      \label{sec:midiamultimidia}

      Os anos de 1999 e 2000 compreendem o conhecido 'marco inicial da primeira década digital'.
      Embora seja principalmente uma forma de delimitar um processo-chave já em andamento, a este período é atribuído
      a delineação dos contornos da 'sociedade em rede'. Esta mudança paradigmática da nossa
      sociedade é sentida agora em maior intensidade nos negócios, na comunicação e na cultura. Estes dois
      últimos fatores são cruciais para o trabalho aqui apresentado dada a notoriedade do papel do
      fenômeno sonoro para a comunicação e expressões culturais. Mais recentemente, podemos falar claramente
      do papel do código nestas duas frentes.

      O termo 'convergência das mídias' não é utilizado para indicar algum dispositivo ou produto que
      lida com várias mídias. Para isso utiliza-se o termo multimídia, que é tema do próximo tópico.
      A convergência das mídias se refere à mundaneidade no lidar com suportes diferentes. Procura-se
      sobre um assunto e lidamos com imagens, textos, videos, áudios e até outras formas diferenciadas
      de mídia, como o código de liguagem de programação.


      HTML5 CSS3 e Javascript deixando interativo. 3D, conexões rápidas que possuem banda para video HD e
      baixa latência. Popularização de hardware capaz destes processamentos.

      Tendência:
      \begin{itemize}
	  \item intensa utilização de audiovisual em contraposição ao
      anterior predominio de textos simples.
	  \item a utilização de periféricos para resultar saída diferenciada - p.ex.
      impressora - ou entrada especializada - p.ex. touch pad.
      \end{itemize}


	  \subsubsection{Gargalos na interatividade e no estado do software}
	  \label{sec:gargalos}
	  Deficiencias em disponibilização da tecnologia atual: muita
	  tecnologia fechada.

	  Comparação do software proprietário com o livre para áudio:

	  XXXXXXXXXXXXXXXXXXXXX

	  Citação da parte de video da thread 'Abrindo o Código Fechado'.

	  Para certas utilizações, a latência não permite a interatividade como 
	  a obtida em presença do objeto examinado seja ele físico ou apenas um BD.
	  A medida do 'agora' é de até 50 ms para humanos [Roads, Microsound] e a latência entre comunicadores é
	  quase sempre maior que isso [tabela de latências, média, por distância, etc].
	  Contribuimos na frente do software. Para isso veja [sessao dos resultados]

	  \subsubsection{Compartilhamento}
	  \label{sec:comp_tec}
	  Terreno fértil -> semear que a primazia do compartilhamento se encarrega de
	  desencadear os desenvolvimentos ligitimos.

	  \subsubsection{Comparacao das soluções em software livre e proprietários}
	  \label{sec:sl_prop}
      [ESCREVER NO FINAL E ABORDAR BREVEMENTE AS QUESTOES DE VIDEO]



\section{Objetivos}
\label{sec:objetivos}
Aqui analisamos os objetivos deste trabalho. Eles refletem os objetivos
dos empreendimentos de SL em geral e tambem das investidas no audiovisual e nas artes.

  \subsection{Difusão de tecnologias e práticas}
  \label{sec:tutoriais}
    \subsubsection{Tutoriais}
    \label{sec:uso_sl}

\begin{itemize}
    \item Tutoriais para programação em python.
Screen casts e texto.

    \item Tutoriais para fazer plugins.
\end{itemize}

  \subsection{Extensão de recursos em software livre para multimídia}
  \label{sec:extensao}

  \subsubsection{Manipulação e tratamento de áudio}
  \label{sec:manip-audio}

  
  \subsubsection{Síntese de sons e estruturas musicais}
  \label{sec:sintese}

  \subsubsection{Capacitação para performance em tempo real e análise}
  \label{sec:perf}
  \label{sec:analise}
