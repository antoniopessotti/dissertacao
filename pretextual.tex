\capa

% \pretextualchapter{}

\folhaderosto
% 
\pretextualchapter{}
	\thispagestyle{plain}
	\noindent \parbox{5.7in}{\centering AUTORIZO A REPRODUÇÃO E DIVULGAÇÃO TOTAL OU PARCIAL DESTE TRABALHO, POR QUALQUER MEIO CONVENCIONAL OU ELETRÔNICO, PARA FINS DE ESTUDO E PESQUISA, DESDE QUE CITADA A FONTE.}

\pretextualchapter{}



%\afterpage{\blankpage}
%
%\pretextualchapter{DEDICATÓRIA} %Título da página dedicatória
%	%Dedico este trabalho às redes civis dedicadas ao livre fluxo de informação. As tecnologias utilizadas neste trabalho são 
%
%Dedico este trabalho à minha família e às pessoas próximas, especialmente
%à minha mulher Thaís Teixeira Fabbri e ao meu filho Antônio Anzoategui Fabbri.







%Dedico este trabalho ao imaterial compartilhado.

\vspace{2cm}
\begin{center}
Dedico este trabalho aos compartilhados imateriais.
\end{center}

 %Para incluir o texto de dedicatória,use o arquivo dedicatória.tex

\afterpage{\blankpage}


\pretextualchapter{AGRADECIMENTOS} %Título da página de agradecimentos
	

Agradeço ao Prof. Luciano da Fontoura Costa e ao Prof. Osvaldo Novais de Oliveira Junior pela oportunidade e pela orientação deste trabalho.

\vspace{4 mm}

Agradeço ao corpo de funcionários do IFSC, pela prestatividade e eficiência em todos os momentos que precisei.

\vspace{4 mm}

Agradeço aos Prof. Rafael Santos Mendes e Prof. Adolfo Maia Junior pelas orientações na gênese deste trabalho.

\vspace{4 mm}

Agradeço à minha família, em especial à minha esposa Thaís Teixeira Fabbri e ao meu filho Antônio Anzoategui Fabbri.

\vspace{4 mm}

Agradeço ao meu irmão Ricardo Fabbri e ao amigo Vilson Vieira da Silva Junior pelas profícuas colaborações acadêmicas, artísticas e socialmente engajadas.


\vspace{4 mm}

Agradeço aos participantes do labMacambira.sf.net pelas colaborações diretas e indiretas nos software, apresentações e articulações desde junho de 2011. Em especial agradeco ao Daniel Penalva, Caleb Macarenhas, Chico Simões, Fábio Simões, Daniel Marostegan, Marcos Mendonça, Geraldo Magela, Patrícia Ferraz, Glerm Soares, Vanessa Ferreira, Sérgio Teixeira de Carvalho. Agradeço aos demais colaboradores das listas de email, IRC, AA e outros canais.


\vspace{4 mm}

Agradeço ao Cultura Viva e aos pontos de cultura pelo apoio fundamental a este e outros trabalhos. Agradeço em especial a estes Pontões e Cultura Digital e Ação Griô pelo suporte a este trabalho em ações regionais e nacionais: CDTL (PE), JuntaDados.org (BA), Nós Digitais (SP), Casa dos Meninos (SP), Nina Griô (SP), Pontão da Eco (RJ), Casa de Cultura Tainã (SP) e à Comissão Nacional dos Pontos de Cultura - CNPdC (GO).


\vspace{4 mm}

Agradeço às comunidades de cultura e software livre por todos os conhecimentos e tecnologias repassados e que compõem esta contribuição. 

%
%
%
%Em primeiro lugar, agradeço aos meus pais e demais familiares. Agradeço aos
%meus amigos pelo carinho e companhia. Aos professores por tanta atenção e compreensão.
%Aos colegas colecionadores de entendimentos por ter-nos construído até aqui.
%Agradeço a Felipe Machado pela visão, a Fabiana 'Goa' Sherine pelo entendimento,
%a Glerm Soares pela transcendência, a Fabianne Baveldi pela excelência, dentre tantos outros que não há páginas para escrever aqui. Agradeço ao profícuo Vilson Vieira, que tornou-se especialmente próximo
%das contribuições citadas neste trabalho, construindo sobre elas e com elas. Condizente com a fortíssima referência
%para que é para mim, tanto para os assuntos tratados nesta dissertação quanto para questões humanas, um especial agradecimento
%ao meu irmão Ricardo Fabbri que também se aproximou destas contribuições tanto cuidando de processamento de imagens e video quanto articulando e amadurecendo as ideias. Agradecimento
%forte ao LabMacambira.sf.net. Um profundo agradecimento
%aos meus orientadores de IC: Adolfo Maia Junior e Jônatas Manzolli. Agradeço a José Augusto Mannis
%pelos trabalhos de estudante, pelo acompanhamento e ensinos sobre manipulação de áudio e música como idioma.
%Agradeço fortemente ao Prof. Dr. Rafael Santos Mendes pela pesquisa em Wavelets de mestrado em engenharia elétrica
%praticamente completa na FEEC/Unicamp, titulo este que não defendendi por razões circunstanciais. Agradeço aos professores da FEEC pelos ensinos de
%primeiríssima qualidade em diversas disciplinas como processamento de sinais, teoria da informação, 
%circuitos elétricos, processos estocásticos
%e programação bio-inspirada. Sem sombra de dúvida, os maiores agradecimentos para os assíduos mentores
%deste trabalho: ao orientador-colaborador Prof. Luciano da Fontoura Costa e ao meu 
%orientator Prof. Osvaldo Novais de Oliveira Júnior. Ambos proporcionaram inestimável acompanhamento e notável paciência, dedicação e precisão em todos os momentos.
 %Para incluir o texto de agradecimentos,use o arquivo agradecimentos.tex


\pretextualchapter{}
	\begin{epigrafetop}
		{Music is a hidden arithmetic exercise of the soul, \\ which does not know that it is counting.}
        {Gottfried Leibniz (1646-1716)}
	\end{epigrafetop}



% \,      a small space
% \:      a medium space
% \;      a large space
% \quad   a really large space
% \qquad  a huge space
% \!      a negative space (moves things back to the left) 
%	\begin{epigrafemid}
%		{They must find it difficult ... \\ Those who have taken authority as the truth, \\ rather than truth as the authority.} %frase
%		{Gerald Massey (1828 - 1907)} %referência
%	\end{epigrafemid}
	\begin{epigrafemid}
		{Music is a hidden metaphysical exercise of the soul, \\ which does not know that it is philosophizing.} %frase
		{Arthur Schopenhauer (1788-1860)} %referência
	\end{epigrafemid}
	\vspace{-1cm}
%	\begin{epigrafebot}
%		{Electronic music used pure sounds, completely calibrated. You had to think digitally, as it were, in a way that allowed you to extend serial ideas into other parameters through technology.} %frase
%		{Luc Ferrari (1929 - 2005)} %referência
%	\end{epigrafebot}

	\begin{epigrafebot}
		{from Pantheon import Obatalá\, \, \, \\
                from World import sharing, kitten \\
                \vspace{.1in}
                while True: \; \; \; \; \; \; \; \; \; \; \; \; \; \; \; \,\\
                    if sharing == 0: \; \; \; \; \;  \; \; \; \; \; \\
                        Obatalá.kill( kitten )}
        {Autoria coletiva e anônima (2010)}
	\end{epigrafebot}



\afterpage{\blankpage}

\resumoeabstract %Para incluir o texto do resumo e abstract,use o arquivo resumoeabstract.tex

\afterpage{\blankpage}


\listadefiguras %Comando que gera a lista de figuras (AUTOMÁTICO)

%\listadetabelas %Comando que gera a lista de tabelas (AUTOMÁTICO)

\afterpage{\blankpage}
\pretextualchapter{LISTA DE RELAÇÕES ANALÍTICAS DESCRITAS E IMPLEMENTADAS COMPUTACIONALMENTE NO APÊNDICE~\ref{cap:codigoProc}}
	\begin{listaespecial}[BIGNAMEWIDTH]
        \item[Equação~\ref{eq:dur}] Sequência de amostras de um áudio PCM.
        \item[Equação~\ref{eq:potencia}] Potência.
        \item[Equação~\ref{decibels}] Decibels entre dois áudios.
        \item[Equação~\ref{eq:ampVol}]  Amplitude dobrada em decibels.
        \item[Equação~\ref{eq:potVol}] Potência dobrada em decibels.
        \item[Equação~\ref{eq:dobraVol}] Variação de amplitude no volume dobrado (10 dBs).
        \item[Equação~\ref{ampDec}] Conversão de decibels em variação de amplitude.
        \item[Equação~\ref{periodicidade}] Representação básica de uma sequência periódica.
        \item[Equação~\ref{senoide}] Sequência infinita senoidal.
        \item[Equação~\ref{denteDeSerra}] Sequência infinita de uma onda 'dente de serra'.
        \item[Equação~\ref{triangular}] Sequência infinita de uma onda 'triangular'.
        \item[Equação~\ref{quadrada}] Sequência infinita de uma onda 'quadrada'.
        \item[Equação~\ref{sampleandoFormaDeOnda}] Sequência infinita de uma forma de onda \emph{sampleada}.
        \item[Equação~\ref{recomposicaoFourier}] Recomposição das amostras temporais com base nos coeficientes espectrais.
        \item[Equação~\ref{moduloEfase}] Recomposição das amostras temporais reais em termos do módulo e fase.
        \item[Equação~\ref{coefsPareados}] Número de pares de coeficientes relativos à mesma frequência.
        \item[Equação~\ref{equivalenciasFreqs}] Coeficientes equivalentes em termos das frequêcias que representam.
        \item[Equação~\ref{equivalenciasModulos}] Equivalências dos módulos dos coeficientes espectrais.
        \item[Equação~\ref{equivalenciasFases}] Equivalências das fases dos coeficientes espectrais.
        \item[Equação~\ref{eq:reconsCompleta}] Reconstrução das amostras temporais em termos dos coeficientes pareados e independentes.
        \item[Equação~\ref{eq:notaBasica}] Nota básica com duração, frequência e altura.
        \item[Equação~\ref{periodoUnico}] Forma de onda sintética ou amostrada.
        \item[Equação~\ref{eq:notaBasicaTimbre}] Nota básica com forma de onda especificada.
        \item[Equação~\ref{eq:distOuvidos}] Distância de uma fonte sonora a cada ouvido.
        \item[Equação~\ref{eq:dti}] Diferença de Tempo Interaural (DTI).
        \item[Equação~\ref{eq:dii}] Diferença de Intensidade Interaural (DII).
        \item[Equação~\ref{eq:locImpl}] Sequência binaural PCM com  DTI e DII para localização espacial.
        \item[Equação~\ref{eq:angulo}] Ângulo resolvido pela implementação da DTI e DII.
        \item[Equação~\ref{eq:mixagem}] Mixagem de N sequências.
        \item[Equação~\ref{eq:concatenacao}] Concatenação de N sequências.
        \item[Equação~\ref{eq:lut}] Procedimento de busca em tabelas (\emph{Lookup Table}).
        \item[Equação~\ref{freqLinear}] Frequências em cada amostra em uma variação linear.
        \item[Equação~\ref{indiceLinear}] Índices para busca na tabela em uma variação linear de frequência.
        \item[Equação~\ref{serieAmostralLin}] Sequência amostral em uma variação linear de frequências.
        \item[Equação~\ref{freqExponencial}] Frequências em cada amostra em uma variação exponencial.
        \item[Equação~\ref{indiceExponencial}] Índices para busca na tabela em uma variação exponencial de frequência.
        \item[Equação~\ref{serieAmostralLog}] Sequência amostral em uma variação exponencial de frequências.
        \item[Equação~\ref{seqAmp}] Sequência de amplitudes em uma variação exponencial.
        \item[Equação~\ref{transAmp}] Sequência amostral em uma variação exponencial de amplitude.
        \item[Equação~\ref{seqAmpLin}] Sequência de amplitudes em uma variação linear de amplitude.
        \item[Equação~\ref{seqAmpDB}] Sequência amostral em uma variação exponencial de amplitude dada em decibels.
        \item[Equação~\ref{eq:conv}] Convolução de sequências reais e finitas.
        \item[Equação~\ref{eq:diferencas}] Equação a diferenças para aplicação temporal de filtros IIR.
        \item[Equação~\ref{eq:passa-baixas}] Coeficientes de um filtro passa-baixas bem comportado de primeira ordem com frequência de corte variável.
        \item[Equação~\ref{eq:passa-altas}] Coeficientes de um filtro passa-altas bem comportado de primeira ordem com frequência de corte variável.
        \item[Equação~\ref{eq:varAux}] Variáveis auxiliares de um filtro nó de segunda ordem.
        \item[Equação~\ref{eq:passa-banda}] Coeficientes de um filtro passa-banda de segunda ordem com frequência central e largura de banda variáveis.
        \item[Equação~\ref{eq:rejeita-banda}] Coeficientes de um filtro rejeita-banda de segunda ordem com frequência central e largura de banda variáveis.
        \item[Equação~\ref{eq:branco}] Coeficientes espectrais para síntese de ruído branco.
        \item[Equação~\ref{eq:rosa}] Coeficientes espectrais para síntese de ruído rosa.
        \item[Equação~\ref{eq:marrom}] Coeficientes espectrais para síntese de ruído marrom.
        \item[Equação~\ref{eq:azul}] Coeficientes espectrais para síntese de ruído azul.
        \item[Equação~\ref{eq:violeta}] Coeficientes espectrais para síntese de ruído violeta.
        \item[Equação~\ref{eq:preto}] Coeficientes espectrais para síntese de ruído preto.
        \item[Equação~\ref{vbrGamma}] Índices auxiliares para um vibrato de frequência variável.
        \item[Equação~\ref{vbrAux}] Expoentes auxiliares para um vibrato de frequência variável.
        \item[Equação~\ref{vbrF}] Frequências por amostra de um vibrato de frequência e profundidade variáveis.
        \item[Equação~\ref{vbrGamma2}] Índices um vibrato de frequência e profundidade variáveis.
        \item[Equação~\ref{vbrT}] Amostras de um áudio com vibrato de frequência e profundidade variáveis.
        \item[Equação~\ref{trA}] Amplitudes por amostra de um tremolo de frequência e profundidade variáveis.
        \item[Equação~\ref{trT}] Amostras de um áudio com tremolo de frequência e profundidade variáveis.
        \item[Equação~\ref{eq:fmEsp}] Espectro da síntese FM.
        \item[Equação~\ref{eq:amEsp}] Espectro da síntese AM.
        \item[Equação~\ref{fmGammaAux}] Índices auxiliares para síntese FM.
        \item[Equação~\ref{fmAux}] Sequência auxiliar para síntese FM.
        \item[Equação~\ref{fmF}] Sequência de frequências para cada amostra para síntese FM.
        \item[Equação~\ref{fmGamma}] Índices para síntese FM.
        \item[Equação~\ref{fmT}] Sequência amostral resultante da síntese FM.
        \item[Equação~\ref{amA}] Sequência de amplitudes por amostra na síntese AM.
        \item[Equação~\ref{amT}] Sequência amostral resultante da síntese AM.
        \item[Equação~\ref{eq:vinculos}] Exemplo de vínculo entre a frequência e parâmetros do tremolo e do vibrato em um som.
        \item[Equação~\ref{eq:adsr}] Envoltória ADSR com variações lineares e logarítmicas.
        \item[Equação~\ref{eq:adsrApl}] Aplicação da envoltória ADSR em uma sequência arbitrária.
        \item[Equação~\ref{eq:intervalos}] Intervalos musicais em número de semitons.
        \item[Equação~\ref{escSim}] Escalas simétricas na oitava divida em 12 notas.
        \item[Equação~\ref{eq:escalas}] Escalas diatônicas.
        \item[Equação~\ref{eq:relacaoDia}] Sucessão de intervalos em uma escala diatônica.
        \item[Equação~\ref{eq:escalasMenores}] Escalas menores natural, harmônica e melódica.
        \item[Equação~\ref{triades}] Tríades maior, menor, diminuta e aumentada.
        \item[Subseção~\ref{subsec:intervalos}] Relações microtonais através de frações de semitons ou através de quantidades inteiras de divisões arbitrárias da oitava.
        \item[Subseção~\ref{subsec:harmonia}] Relações básicas de harmonia tonal.
        \item[Subseção~\ref{subsec:contraponto}] Regras básicas de condução de vozes com independência.
        \item[Subseção~\ref{subsec:ritmo}] Relações básicas de métrica e rítmica.
        \item[Subseção~\ref{subsec:dir}] Formação de arcos musicais através de estruturas direcionais.
        \item[Subseção~\ref{estCic}] Estruturas cíclicas para a síntese musical.
%        \item[Equação~\ref{eq:groups}] Propriedades básicas de um grupo algébrico.

	\end{listaespecial} 


\pretextualchapter{Elementos de notação utilizados na dissertação}
	\begin{listaespecial2}[BIGNAMEWIDTH]
        \item[$Hz$] abreviação de Herz, medida de frequência, número de ocorrências por segundo.
        \item[$kHz$] abreviação de kilo Hertz, i.e. mil Herz.
        \item[$m$] metro.
        \item[$mm$] milimetro.
        \item[$s$] segundo.
        \item[$m/s$] metros por segundo.
        \item[$f_a$] frequência de amostragem, taxa de amostragem.
        \item[$\lambda_a$] duração da separação temporal entre um par de amostras consecutivas.
        \item[PCM] sigla de \emph{Pulse Code Modulation}, veja seção~\ref{sec:audio}
        \item[$S_i$] sequência $S$ indexada em $i$.
        \item[$S_i=\{s_i\}_x^{y}$] sequência $S_i$ com elementos $s_i$com índices de $x$ a $y$ incluso.
        \item[$\lfloor x \rfloor$] parte inteira de $x$.
        \item[$V_{dB}$] volume em decibels.
        \item[$\Rightarrow$] então.
        \item[$\therefore$] portanto.
        \item[$\approx$] aproximadamente.
        \item[$\equiv$] equivalente.
        \item[$\%$] operação módulo, resto da divisão.
        \item[$\Lambda$] comprimento em amostras.
        \item[$\widetilde{\Lambda}$] comprimento em amostras de período de onda.
        \item[$\Delta$] duração temporal.
        \item[$a_i$] fator multiplicativo de amplitude da $i$-ésima amostra.
	\end{listaespecial2} 


%
%\afterpage{\blankpage}
%
%\pretextualchapter{Lista de peças musicais}
%	\begin{listaespecial}[BIGNAMEWIDTH]
%		\item[Chorus infantil]: relações harmônicas da seção~\ref{}. Música disponibilizada online~\cite{}. Arquivo Python no apêndice~\ref{}.
%	\end{listaespecial} 
%
%\afterpage{\blankpage}
%\pretextualchapter{Lista de Abreviaturas}
%	\begin{listaespecial}[BIGNAMEWIDTH]
%		\item[OSS] Programas de Código Aberto \emph{(Open Source Software)}
%		\item[GNU] \emph{GNU is Not UNIX}
%		\item[GPL] \emph{General Public Licence}
%                \item[AA] \emph{Algorithmic Autoregularion ou Acrônimo Ambíguo}
%                \item[AC] \emph{Autogestão Coletiva ou Ágora Communs}
%                \item[CC] \emph{Creative Commons}
%                \item[ABT] \emph{ABeatTracker}
%                \item[ABD] \emph{ABeatDetector}
%                \item[PD] \emph{Pure Data}
%                \item[CMDCA] \emph{Conselho Municipal de defesa dos Direitos da Cirança e do Adolescente}
%                \item[SOS] \emph{Saúde Olha Sabedoria: uma proposta de coleta e difusão de conhecimentos relacionados à saúde}
%                \item[LADSPA] \emph{Linux Audio Developers Simple Plugin API}
%                \item[LV2] \emph{LADSPA Version 2}
%                \item[JACK] \emph{Jack Audio Connection Kit}
%                \item[SIP] \emph{Scilab Imaging Processing toolbox}
%                \item[MIT] \emph{Massachusetts Institute of Technology}
%                \item[NUMPY] \emph{Numerical Python}
%                \item[SCIPY] \emph{Scientific Python}
%                \item[ONU] \emph{Organização das Nações Unidas}
%                \item[SL] \emph{Software Livre}
%                \item[EKP] \emph{Emotional Kernel Panic}
%                \item[SOS] \emph{Saúde Olha Sabedoria}
%                \item[FISL] \emph{Festival Internacional de Software Livre}
%                \item[BPM] \emph{Batidas Por Minuto (medida de andamento musical)}
%                \item[API] \emph{Application Programming Interface ou Interface de Programação de Aplicativos}
%                \item[WP] \emph{Wavelet Packet}
%
%		\item[EL] \emph{Estúdio Livre}
%		\item[AE] \emph{AudioExperiments}
%		\item[PCM] \emph{Pulse Code Modulation (modulação por código de pulsos)}
%		\item[DTI] \emph{Diferença de Tempo Interaural (\emph{Interaural Time Difference} - ITD)}
%		\item[DII] \emph{Diferença de Intensidade Interaural (\emph{Interaural Intensity Difference)} - IID ou \emph{Interaural Level Difference} - ILD}
%	\end{listaespecial} 
%

\sumario

