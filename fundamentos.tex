%% ------------------------------------------------------------------------- %%
\chapter{Fundamentos teóricos e técnicos} %Nome do capítulo.
% \label{cap:intro} %Rótulo para futura referência ao capítulo. Em qualquer lugar da tese, você poderá citar este capítulo através de ~\ref{cap:introducao}. Você escolhe o argumento de \label e pode ser qualquer coisa (Ex: \label{Procedimento_Experimental})

\section{Áudio}

  \subsection{O som e a audição}

O som é o fenômeno físico de propagação de ondas mecânicas. Ele pode
somente ocorrer na presença de matéria, como ar, água, ferro ou madeira.
São ondas de pressão que se deslocam com velocidade caracterizada principalmente
pelo meio material e também pelas condições como a temperatura e a pressão.
O vácuo é um isolante acústico perfeito.

Tipicamente a velocidade do som é de $~340 m/s $ e é
relativamente estável\footnote{$343.2 m/s$ em ar seco a 20 °C (68 °F)}.
As modificações de sonoridades com o clima é devido a alterações
dos materiais que os geram e acústica do ambiente, não devido a alterações
desta velocidade de propagação do som no meio atmosférico.

Falar de fonons e coisas assim? Quais aspectos do som devemos cobrir?

      \subsubsection{A audição e a psicofísica}

A audição é definida com a percepção de vibrações mecânicas com
ou sem a contribuição direta do tato. Em diversos
seres vivos, encontramos um órgão dedicado a captar e transmitir
estas vibrações mecânicas (som) para o sistema nervoso.

O ouvido humano cumpre exatamente este papel. Um esquema básico
do ouvido humano conta com 3 divisões. No ouvido externo está a orelha.
No médio, os martelinhos e o tímpano. No ouvido interno tem a cóclea
e os cíclios. Uma visualização simplificada do ouvido humano está abaixo.


FIGURA MOSTRANDO ENTRADA DE VIBRAÇÃO E TRANSMISSÃO AO CÉREBRO

Vale notar que a audição conta em grande parte com o tato e potencialmente
outros órgãos, como o estômago.

Já a psicofísica busca relacionar características físicas do fenômeno sonoro
com as percepções individuais resultantes\footnote{PsycoPy para escrever experimentos psicofísicos em Python: http://www.psychopy.org/}.
Importantes exemplos destas relações
são a escala logarítimica que a percepção linear das alturas assume no
domínio da frequência e as curvas iso-audíveis (conhecidas como curvas de Fletcher-Münsen).
EXPLICAR MELHOR?

Outro importante aspecto da psicofísica do som - e da música - é a resolução da percepção
e a capacidade de notar características do espectro em contextos específicos. O mascaramento
de frequências é a base na qual se faz compactadores de tamanhos de arquivos de áudio. Sabe-se
que, por exemplo, no extremo grave, não conseguimos distinguir duas senóides a quase um tom de
diferença. Alguns outros importantes fenômenos psicoacústicos estudados com fins tecnológicos, científicos e artísticos são: espacialidade,
relação da percepção de volume com espectro agudo e reverberação e compreensibilidade de sinais de fala.


\subsection{Áudio digitalizado}
Enquanto estamos interessados na variação de pressão do ar, podemos de fato utilizar
um microfone para converter esta variação de pressão em sinal elétrico (usualmente voltagem).
Convertemos assim o som (fenômeno físico) em áudio (sinal elétrico relacionado).
Neste processo, o sinal é amostrado no tempo geralmente através do uso de
um conversor A/D\footnote{Conversor Analógico para Digital. Para as aplicações descritas aqui é tipicamente uma placa som.}. Assim, um evento sônico
torna-se um sinal digital, usufruindo das facilidades para armazenamento, transmissão e
processamento da nossa tecnologia digital atual.

Esta amostragem do sinal de áudio é bastante elaborada e dependente de manejo de hardware. Para
aprofundar este tema, recomendamos [][][] e uma representacao esquematica deste processo
se encontra na figura.	 
% \figure{fig:AD}.

FIGURA MOSTRANDO UMA ONDA ANALOGICA E ELA REPRESENTADA POR PONTOS.

A representação digital do som pode ser feita de diversas formas.
A representação PCM (Pulse Code Modulation) canônica possui amostras
igualmente espaçadas no tempo e para cada uma utiliza precisão com o mesmo
número de bytes.

[Figura com onda amostrada com limites superior inferior e valores das amostras]

Note que o áudio digital possue também uma velocidade no espaço de amostragem.
Quantidades como $8kHz$, $44.1kHz$, $48kHz$
descrevem quantidade de amostras em cada segundo com que o áudio examinado foi feito ou
está sendo utilizado.

O importante Teorema de Nyquist demonstra que em um sinal digital só se consegue representar
frequências até metade da taxa de amostragem. Ou seja, em um áudio com $48kHz$ de taxa de amostragem,
conseguimos representar frequências de até $24kHz$. A frequência de Nyquist é exatamente a metade
da frequência de amostragem e é a frequência mais aguda que aquele sinal pode representar.
(PROVA ELEGANTE DO TEOREMA DE Nyquist? TAMB REFERENCIAR OS SCRIPTS PARA ESTUDO DO FENOMENO)

Além da representação PCM, pode-se utilizar outros formatos para o armazenamento do áudio. Alguns
destes formatos apresentam perda de qualidade e são mais indicados para serem transmitidos via rede ou
para armazenamento\footnote{Os formatos que apresentam perda de qualidade com relação
ao PCM são chamados formatos \emph{lossy}, os que não apresentam perda de qualidade
são chamados \emph{loseless}}. Abaixo está uma tabela com alguns dos formatos mais importantes que possuímos:

TABELA DE FORMATOS


\subsection{Processamento de áudio}

Síntese, alteração ou análise.

\begin{itemize}
    \item Geral
    \item Voz
    \item Música
\end{itemize}

Procedimentos básicos: Lookuptable, sample. Processamento
espectral. Outros processamentos.

Vincular a um apêndice com excertos de código.

\section{Disponibilização da tecnologia em código}

  \subsection{Python}


  \subsection{Git}

