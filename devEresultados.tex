%% ------------------------------------------------------------------------- %%
\chapter{Desenvolvimentos e resultados} %Nome do capítulo.
\label{cap:resultados} %Rótulo para futura referência ao capítulo. Em qualquer lugar da tese, você poderá citar este capítulo através de ~\ref{cap:introducao}. Você escolhe o argumento de \label e pode ser qualquer coisa (Ex: \label{Procedimento_Experimental})

\section{Materiais didáticos}

\subsection{Tutoriais em texto e código}

\begin{itemize}
    \item Tutorial de python para áudio e som.

    \item Tutoriais de filtros e amostragem via python.

    \item Tutorial de plugins lv2

    \item Microtutoriais Django ~\cite{dmicrotuts}.

    \item Philosometrics

    \item Carta mídias livres

    \item Textos de cunho sociológico, transformador

de      fabi borges catadores@gmail.com por  riseup.net 
responder a     submidialogia@lists.riseup.net
para    submidialogia@lists.riseup.net
data    23 de agosto de 2011 12:26
assunto Re: [submidialogia] livro sub- publicação
lista de e-mails        <submidialogia.lists.riseup.net> Filtrar as mensagens dessa lista de e-mails
enviado por     lists.riseup.net
assinado por    riseup.net
cancelar inscrição      Cancelar a inscrição para essa lista de e-mails
        Importante principalmente porque você frequentemente lê mensagens com esse marcador.
ocultar detalhes 12:26 (6 minutos atrás)
entao, eu fui recebendo textos durante esse tempo,
alguns tao atrazados como dos sem satelites, mas muita gente mandou;

aqui os autores:

os internacionais nao sao muitos, o joni kempf (o que bebe ouro do hardware), o barbrook (futuros imaginarios),
Hamdy heda (da revolucao egipcia), o pedro soller (summerlab), maria llopis (pos porno),  talvez a bronac, nao entregou ainda.

dos brasileiros, renato fabri, ruiz, pasteur, morgana e caio, ju dornelles, mari marcassa, coletivo errorista, adriana velozo-drica, tiago pimentel, felipe fonseca, thiago novaes, lelex, felipe ribeiro(?), maira, verenilde,  bartolina silva, poro, vitoria amaro, fabib (eu),

entao, precisa publicar agora,
uma equipe para publicacao e,,, 

bjs
f


\end{itemize}

\subsection{Screencasts}

\begin{itemize}
    \item Python para áudio e música

    \item Canal Macambira
No coletivo Macambira estão sendo produzidos materiais em screencasts sobre
diversas.
    \begin{itemize}
	\item Live-Coding
	\item Raspagem de dados
    \end{itemize}
\end{itemize}



\subsection{Figusdevpack (FDP)}

Um ambiente de interação da comunidade de Python e Música 
para compartilhamento de códigos e excertos. Baseado principalmente
em documentação organizada sobre as práticas e as bibliotecas
existentes para python. Assim como scripts compartilhados que
fazem uso de objetos e módulos específicos. Este trabalho foi aceito na
maior conferência de áudio em linux, a Linux Audio Conference de 2008
(LAC2008) e está sendo reativado por mim em conjunto com Vilson Vieira
e outros desenvolvedores de áudio. Este projeto está em desenvolvimento
do site do Estúdio Livre [estudio livre] com repositório no sourceforge [source force].

As principais fontes sobre estre trabalho é a página de desenvolvimento da ideia
que está em constante mudança ~\cite{http://estudiolivre.org/tiki-index.php?page=fdp&highlight=fdp fdpel}
e, o artigo que foi aceito no LAC de 2008 ~\cite{http://www.estudiolivre.org/el-gallery_view.php?arquivoId=8221 fdplac2008}
e o repositório ~\cite{http://sourceforge.net/projects/fdpack/develop fdpsf}.

\section{Áudio e Música}

\subsection{ABeatTracker (ABT)}

Manual ABT

Linkar com ABD

\subsection{Emotional Kernel Panic (EKP)}

Em 2008, colaborando intensamente com o CDTL
(Centro de Desenvolvimento de Tecnologias Livres)\footnote{foi uma associação civil formada e desmembrada em 2008 e sediada em Recife, PE}
foi lançada a ideia de utilizar o estado do sistema operacional - especialmente o kernel linux - para
geração de sons. Surge o Emotional Kernel Panic (EKP). Desde o inicio, foram definidos três finalidades
para esta exploração do SO:

\begin{itemize}
    \item Didáticos
    \item Artísticos
    \item Monitoramento do SO
\end{itemize}

http://trac.assembla.com/audioexperiments/browser/ekp-base


\subsection{Finite Groups in Granular Synthesis (FIGGS)}

\subsection{Plugins LADSPA e lv2}

\subsection{Protocolo de compressão de áudio via wavelets, polinômios e permutações}

\subsection{Processamento de voz via redes complexas}

\subsection{AirHackTable}

\subsection{Experimentos abertos}

Audioexperiments. Estúdio livre.

\section{Web}

Difusão de informação com ênfase na facilitação
da apropriação de tecnologias e de instancias políticas.

\subsection{Tecnologias sociais de alta demanda}

\subsection{Sítios}

FDDCA

Ferramenta de comunicação

(Cadastro dos pontos?)

AA, SOS, Catalogo de Ideias, etc

Meu site pessoal


\subsection{Conteúdos}

Wiki?

\subsection{Articulação}

IRC, Emails

\section{Disponibilização e Desenvolvimento}

\subsection{Wiki}

\subsection{Trac}

\subsection{Screencasts - Vimeo}

\subsection{AA}

\subsection{Audio Experiments (Æ)}

\subsection{IRC}

\subsection{Etherpads}

\subsection{Outras fontes}