%% ------------------------------------------------------------------------- %%
\chapter{Desenvolvimentos e resultados} %Nome do capítulo.
\label{cap:resultados} %Rótulo para futura referência ao capítulo. Em qualquer lugar da tese, você poderá citar este capítulo através de ~\ref{cap:introducao}. Você escolhe o argumento de \label e pode ser qualquer coisa (Ex: \label{Procedimento_Experimental})

\section{Materiais didáticos}

\subsection{Tutoriais em texto e código}

Os vários materiais didáticos produzidos constam no apêndice
deste trabalho. Um único destes será exposto a seguir por sua
capacidade de agregar os conteúdos dos capítulos anterioes:
os tutoriais de filtros e amostragem.

\begin{itemize}
    \item {\bf Tutorial de python para áudio e som}

Este tutorial foi levado para Berlim no LAC 2007 e sofreu melhoras desde entao. Esta
primeira versao ficou resumida em forma de texto no EL\footnote{http://estudiolivre.org/python-e-som-tutorial}. Em 2010
a Associacao Python Brasil escolheu este trabalho, então já mais amadurecido, para ser apresentado no
FISL em Porto Alegre. Como consequencia, foi feita uma série de video-tutoriais bastante utilizados\footnote{http://estudiolivre.org/tiki-index.php?page=Video+Tutoriais}.
Este tutorial foi comentado em listas em que o autor não participa (e outras em que o autor participa).

    \item {\bf Tutoriais de filtros e amostragem via python}

Voltados para explicitar principios fundamentais de áudio, estes tutoriais
são baseados código Python e o equivalente em C. Pequenas explicações são
dadas com o intuito de orientar a exploração inteligente destes \emph{snippets}.

\emph{Teorema de Amostragem}: estes scripts visam a experimentacao inteligente com
o Teorema de Nyquist. (descricao)

\emph{Filtros}: alem da explicitação sobre as diferenças entre filtros FIR e IIR,
duas utilizações clássicas destes filtros estão implementadas: Wavelets (FIR) e Quad (IIR)

Estes códigos podem ser baixados no repositório SVN do AudioExperiments. E os textos estão
na wiki (nos digitais ou EL, recriar pois os CDTL foram apagados)

    \item {\bf Tutorial de plugins lv2}

Dadas as dificuldades que o desenvolvimento dos \emph{plugins} de áudio apresenta,
desenvolvi um tutorial passo a passo com plugins que rodam em todas as etapas.
Ele é baseado em uma interface C++ para este padrão de plugin que eh implementado
em C. Os códigos e os textos estão todos em repositório.

    \item {\bf Microtutoriais Django ~\cite{dmicrotuts}}

Estes 'microtutoriais' são baseados nos conceitos de \emph{scripts mínimos} e
\emph{alterações puntuais}. O primeiro conjunto de microtutoriais é dedicado
a reconstruir o tutorial oficial do django de forma condensada e não prolixa.
O segundo destes conjuntos é dedicado a instrumentalizar de fato o leitor com
o entendimento do funcionamento dos princípios fundamentais deste framework.

    \item {\bf Philosometrics}

Embora este não seja um trabalho didático propriamente dito, ele tem este intuito
no cerne de sua concepção e surgimento. Em decorrência dele, surgiu o  Musimetrics,
o Cinemetrics e o Literametrics. Além disso, ele é um belo exemplo da
utilização das ciências duras para a análise de ciências humanas e foi acolhido
como tal em alguns momentos.

    \item {\bf Carta mídias livres}

Texto criado em decorrência da participação da comissão de seleção no
'Prêmio Mídias Livres', a convite do Ministério da Cultura por 'notório saber'.
Esta carta é um documento único no seu conteúdo, deixando às claras
o conceito de Mídias Livres como mídias não aprisionadas pelo conceito
de propriedade, ou seja, que priorizam a sua livre circulação e a possibilidade
de geração de materiais derivados, assim como sua geração aberta ao colaborativo e comunitário.

    \item {\bf Textos de cunho sociológico, transformador}

Produção mais numerosa que as anteriores, se caracteriza por métodos não convencionais
de abordagem dos assuntos e de escrita. Em especial utiliza-se pseudônimos para
auxiliar a despersonificação, gerando textos menos presos à satisfação da auto-imagem, dente
outras qualidades. A utiliação de psudônimos é um costume muito apreciado em diversos meios,
e as pesquisas tem confirmado as vantagens que a prática apresenta e confirma\footnote{http://disqus.com/research/pseudonyms/}.

O autor destes textos se dá ao direito de não revelar seus psudônimos - embora muitos deles
sejam publicamente conhecidos - para conservar as consequências desta prática na
forma mais pura. Como comprovante desta produção, deixamos uma mensagem sobre a publicação
de textos em mídia impressa com autores internacionais,
confirmando a participação do autor desta dissertação, mas cujo
nome não consta na publicação.

!!!!!!!!!!!!!!!!!!!!!!!!!!!!!!!!!!!!!!!

de      fabi borges catadores@gmail.com por  riseup.net 
responder a     submidialogia@lists.riseup.net
para    submidialogia@lists.riseup.net
data    23 de agosto de 2011 12:26
assunto Re: [submidialogia] livro sub- publicação
lista de e-mails        <submidialogia.lists.riseup.net> Filtrar as mensagens dessa lista de e-mails
enviado por     lists.riseup.net
assinado por    riseup.net
cancelar inscrição      Cancelar a inscrição para essa lista de e-mails
        Importante principalmente porque você frequentemente lê mensagens com esse marcador.
ocultar detalhes 12:26 (6 minutos atrás)
entao, eu fui recebendo textos durante esse tempo,
alguns tao atrazados como dos sem satelites, mas muita gente mandou;

aqui os autores:

os internacionais nao sao muitos, o joni kempf (o que bebe ouro do hardware), o barbrook (futuros imaginarios),
Hamdy heda (da revolucao egipcia), o pedro soller (summerlab), maria llopis (pos porno),  talvez a bronac, nao entregou ainda.

dos brasileiros, renato fabri, ruiz, pasteur, morgana e caio, ju dornelles, mari marcassa, coletivo errorista, adriana velozo-drica, tiago pimentel, felipe fonseca, thiago novaes, lelex, felipe ribeiro(?), maira, verenilde,  bartolina silva, poro, vitoria amaro, fabib (eu),

entao, precisa publicar agora,
uma equipe para publicacao e,,, 

bjs
f

!!!!!!!!!!!!!!!!!!!!!!!!!!!!!!!!!!!!!!!

\end{itemize}

\subsection{Screencasts}

\begin{itemize}
    \item Python para áudio e música

    \item Canal Macambira
No coletivo Macambira estão sendo produzidos materiais em screencasts sobre
diversas.
    \begin{itemize}
	\item Live-Coding
	\item Raspagem de dados
    \end{itemize}
\end{itemize}



\subsection{Figusdevpack (FDP)}

Um ambiente de interação da comunidade de Python e Música 
para compartilhamento de códigos e excertos. Baseado principalmente
em documentação organizada sobre as práticas e as bibliotecas
existentes para python. Assim como scripts compartilhados que
fazem uso de objetos e módulos específicos. Este trabalho foi aceito na
maior conferência de áudio em linux, a Linux Audio Conference de 2008
(LAC2008) e está sendo reativado por mim em conjunto com Vilson Vieira
e outros desenvolvedores de áudio. Este projeto está em desenvolvimento
do site do Estúdio Livre [estudio livre] com repositório no sourceforge [source force].

As principais fontes sobre estre trabalho é a página de desenvolvimento da ideia
que está em constante mudança ~\cite{http://estudiolivre.org/tiki-index.php?page=fdp&highlight=fdp fdpel}
e, o artigo que foi aceito no LAC de 2008 ~\cite{http://www.estudiolivre.org/el-gallery_view.php?arquivoId=8221 fdplac2008}
e o repositório ~\cite{http://sourceforge.net/projects/fdpack/develop fdpsf}.

\section{Áudio e Música}

exp abertos

Do produção de códigos em tempo real para música, chamado livecoding,
vamos para ferramentas computacionais para áudio e música, com o minimum-fi
e FIGGUS. Já fazendo ponte do código com outras interfaces, adentramos AHT e EKP,
que interfaceiam com o mundo físico e material (AHT) e com o hardware, onde
nosso entendimento em codigo é efetuado em procedimentos.

  \subsection{Experimentos abertos em áudio: LADSPAs, Wavelets e Redes Complexas}

Todos os desenvolvimentos desta dissertação estão em repositórios abertos\cite{repositorios-tese-dev},
e alguns foram especialmente importantes como percurso para o que é apresentado neste
escrito. Ou seja, 

Focaremos no código para a arte sonora,
incluindo a musical, e portanto fazendo parte da cultura e fazendo cultura.
\footnote{As chamadas culturas biopunk, ciberpunk, cipherpunk, hacker, digital e outras mais,
possibilitadas aos recentes desenvolvimentos em telecomunicação, dizem respeito em menor
ou maior grau ao código.}

A seguir apresentamos alguns bons exemplos dos desenvolvimentos
desta dissertação especificamente em áudio, sem o envolvimento direto da música.

      \subsubsection{Plugins LADSPA e lv2}
      [repos AE, LM, wiki EL, historico CDTL]


      \subsubsection{Protocolo de compactação de áudio via wavelets, polinômios e permutações}
      [achar artigo feec]


      \subsubsection{Processamento de voz via redes complexas}
      [repo do audio experiments. passar para o LM e chamar o TT]
  

  \subsection{Música em Tempo Real: Livecoding e ABeatTracker (ABT)}

  [Apresentar Livecoding.]

  Apenas parte dos conhecedores do assunto reconhecem
  o ABT como livecoding. Alguns
  consideram a linguagem por macros mais especializadas e poderosas como
  sendo um tipo de interface escrita, nao uma linguagem de programação
  propriamente dita.

      \subsubsection{ABeatTracker}


  Manual ABT

  Linkar com ABD

      \subsubsection{Audioexperiments, Estudiolivre e Foobarbaz}

      Audioexperiments. Estúdio livre. LM




  \subsection{Musica em Tempo Diferido: Minimum-fi e FIGGUS}

      \subsubsection{Minimum-fi}

Em Abril de 2011 foi trabalhado um codigo dedicado a materializar o minimo
para se conseguir expressar ideias musicais em geral. Esta empreitada
resultou no minimum-fi, codigo em um unico arquivo, que sintetiza
musicas inteiras. O os algoritmos propriamente ditos somam menos de
50 linhas e nao mais que 5 funções.

Os princípios são:
\begin{itemize}
  \item Deve-se ter um mecanismo de sintese simples mas que
permita varios timbres.
  \item Deve-se ser capaz de construir series de notas e acordes.
\end{itemize}

Para o primeiro item, se prestam as funções lookup e lookupcruz. A função
lookup já foi apresentada na sessão anterior, quando tratado de Python já visando o áudio.
O lookupcruz ex explicado a seguir.

[codigo do lookupcruz e explicacao]

As sequencias de notas e os acordes foram resolvidos com 2 outras funcoes simples,
que utilizam lookup e lookupcruz.

[codigo do fazSequencia e explicacao]

[codigo do fazAcorde e explicacao]

A função somador eh para automatizar a soma amostra a amostra e completar com zeros a
sequencia menor.

[codigo do somador, explicacao e figura de uma sequencia maior/menor que a outra]

Depois disso é usufruir com estruturas criadas. Por exemplo: depois de construidas
algumas tabelas para serem usadas como diferentes timbres, pode-se criar as
escalas completamente simetricas assim:

escala\_1=range(12) # cromática ascendente sem inclusão da oitava
escala\_2=range(0,12,2) # tons inteiros
escala\_3=range(0,12,3) # diminutão
escala\_4=range(0,12,4) # terças maiores
escala\_6=range(0,12,6) # trítonos

Sendo cada unidade um semitom. Assim as triades maiores e menores
são especificadas com simplicidade:

maior=[0,4,7]
menor=[0,3,7]

E series podem ser anotadas junto a variacoes:

b1=[0]*4
b2=[0]*7+[7]
b3=  [0,7,0,1]
b3\_2=[0,7,8,7]
b4=[0,7,7,7]
b5=[0,7]
b6=[0,19,7,12]
b7=[0,-1,0,-1]
b8=[0,-1,7,-1]
b9=[0,1,0,-1]
b10=[0,7,6,7]

Agora eh sintetizar e mixar. A sintese de sequencias e acordes sao feitos tipicamente
atraves de comandos desta forma:

fazSequencia(nt, f, fator, d, s, senoide, dente)
fazAcorde([0,7,14,21,28,35,42,49,56,63,70,77,84]

Aqui uns procedimentos basicos utilizados nas montagens:

seq=[1.5*i for i in fazSequencia(nt, f, fator, d, s, senoide, dente)] # amplificacao

seq3=[(i+j) for i,j in zip(seq1,seq2)] # mixagem

Gerando uma serie de acordes:

ac=[]
for i in xrange(8): # a cada 2 tempos
    ac+=fazAcorde([0,7,14,21,28,35,42,49,56,63,70,77,84], f, fator, d, s, senoide, dente)
    ac+=[0]*int((d+s)*44100)

E por fim, mixando quando os comprimentos nao sao os mesmos:

seq2=somador(ac,seq2)




\subsubsection{Finite Groups in Granular and Unit Synthesis (FIGGUS)}

Desenvolvimento iniciado em 2006 com o físico-matemático Prof. Adolfo Maia Junior para
tratar de simetrias na música, com vistas à composição musical atraves
de métodos matemáticos. Mais especificamente, a proposta resultou em
um programa voltado para o uso de Grupos Algébricos para síntese
granular e síntese de estruturas musicais. O nome dado
foi FIGGUS, sigla de FInite Groups in Granular and Unit Synthesis.

A Síntese Granular é uma área bem estabelecida tanto na acústica quanto
na Computação Musical. O Físico Gabor, quantum sonoro, phonon, etc..
[achar artigos feitos com adolfo]

Na atual reescrita, embora ainda bastante atras do FIGGUS original,
mesmo sem contar a interface grafica, o codigo roda inteiro em python
com as biblitecas imbutidas por padrão. Isso permite com que o FIGGUS
sintetize todo um EP usando somente os comandos:

[[ARRUMAR A PARADA PARA O CODIGO]]
git clone
python setup.py install
python RUNME




  \subsection{Música na Matéria: EKP e AHT}
      \subsubsection{Emotional Kernel Panic (EKP)}

Em 2008, colaborando intensamente com o CDTL
(Centro de Desenvolvimento de Tecnologias Livres)\footnote{foi uma associação civil formada e desmembrada em 2008 e sediada em Recife, PE}
foi lançada a ideia de utilizar o estado do sistema operacional - especialmente o kernel linux - para
geração de sons. Surge o Emotional Kernel Panic (EKP). Desde o inicio, foram definidos três finalidades
para esta exploração do SO:

\begin{itemize}
    \item Didáticos
    \item Artísticos
    \item Monitoramento do SO
\end{itemize}

http://trac.assembla.com/audioexperiments/browser/ekp-base


      \subsubsection{AirHackTable}


\section{Web}

Difusão de informação com ênfase na facilitação
da apropriação de tecnologias e de instancias políticas.

\subsection{Tecnologias sociais de alta demanda}

\subsection{Sítios}

FDDCA

Ferramenta de comunicação

(Cadastro dos pontos?)

AA, SOS, Catalogo de Ideias, etc

Meu site pessoal


\subsection{Conteúdos}

Wiki?

\subsection{Articulação}

IRC, Emails

\section{Disponibilização e Desenvolvimento}

\subsection{Wiki}

\subsection{Trac}

\subsection{Screencasts - Vimeo}

\subsection{AA}

\subsection{Audio Experiments (Æ)}

\subsection{IRC}

\subsection{Etherpads}

\subsection{Outras fontes}