%% ------------------------------------------------------------------------- %%
\chapter{Desenvolvimentos e resultados} %Nome do capítulo.
\label{cap:resultados} %Rótulo para futura referência ao capítulo. Em qualquer lugar da tese, você poderá citar este capítulo através de ~\ref{cap:introducao}. Você escolhe o argumento de \label e pode ser qualquer coisa (Ex: \label{Procedimento_Experimental})


\section{Áudio e Música}

exp abertos

Do produção de códigos em tempo real para música, chamado livecoding,
vamos para ferramentas computacionais para áudio e música, com o minimum-fi
e FIGGUS. Já fazendo ponte do código com outras interfaces, adentramos AHT e EKP,
que interfaceiam com o mundo físico e material (AHT) e com o hardware, onde
nosso entendimento em codigo é efetuado em procedimentos.

  \subsection{Experimentos abertos em áudio: LADSPAs, Wavelets e Redes Complexas}

Todos os desenvolvimentos desta dissertação estão em repositórios abertos\cite{repositorios-tese-dev},
e alguns foram especialmente importantes como percurso para o que é apresentado neste
escrito. Ou seja, 

Focaremos no código para a arte sonora,
incluindo a musical, e portanto fazendo parte da cultura e fazendo cultura.
\footnote{As chamadas culturas biopunk, ciberpunk, cipherpunk, hacker, digital e outras mais,
possibilitadas aos recentes desenvolvimentos em telecomunicação, dizem respeito em menor
ou maior grau ao código.}

A seguir apresentamos alguns bons exemplos dos desenvolvimentos
desta dissertação especificamente em áudio, sem o envolvimento direto da música.

      \subsubsection{Plugins LADSPA e lv2}
      [repos AE, LM, wiki EL, historico CDTL]


      \subsubsection{Protocolo de compactação de áudio via wavelets, polinômios e permutações}
      [achar artigo feec]


      \subsubsection{Processamento de voz via redes complexas}
      [repo do audio experiments. passar para o LM e chamar o TT]
  



  \subsection{Musica em Tempo Diferido: Minimum-fi e FIGGUS}

\begin{quotation}
\small
'The increasing dominance of graphic interfaces for music software obscured 
the continuing presence of the command-line tradition, 
the code writer, the hacker. The code writing of deferred time 
computer programming may be assembled out of time order, debugged and optimized.'

\emph{Simon Emmerson, Living electronic music, 2007}
\end{quotation}

A realização musical em tempo diferido é o paradigma inicial da música computacional.
Iniciando com o Music V, a proposta foi depois desenvolvido com o CSound. Pode-se dizer
que até hoje é a forma como compositores usualmente pensam a música: pensando
e escrevendo as estruturas, que depois são executadas por instrumentistas.

Assim como a composição instrumental permite um trabalho mais minucioso do que
a improvisação instrumental, a realização musical em tempo diferido permite um
detalhamento maior dos procedimentos do que a realização em tempo real. Por este
mesmo motivo, trataremos inicialmente de dois trabalhos em tempo diferido,
aliás paradigmaticos.

O primeiro, chamado de \emph{minimum-fi}\footnote{hi-fi, low-fi, minimum-fi, ou seja,
o mínimo de qualidade para assegurar existência e consistência
da criação} sintetiza uma música inteira, amostra por amostra, através de princípios
claros de síntese sonora e organização musical. O segundo, chamado de \emph{FIGGUS} (FInite
Groups in Granular Synthesis), utiliza os princípios do minimum-fi e constitui
um módulo Python completo, cuja proposta é a utilização de simetrias, através de
permutações e de Teoria de Grupos, para a composição de músicas. Como demonstração
das capacidades do FIGGUS, ele gera um EP inteiro com um único comando. Este é o
\emph{PPEPPS}\footnote{Pure Python EP: Projeto Solvente}, como veremos a seguir.

      \subsubsection{Minimum-fi}

Em Abril de 2011 foi trabalhado um codigo dedicado a materializar o minimo
para se conseguir expressar ideias musicais em geral. Esta empreitada
resultou no minimum-fi, código em um único arquivo, que sintetiza
músicas inteiras. O os algoritmos propriamente ditos somam menos de
50 linhas e nao mais que 5 funções.

Os princípios são:
\begin{itemize}
  \item Deve-se ter um mecanismo de sintese simples mas que
permita varios timbres.
  \item Deve-se ser capaz de construir unidades e series de unidades (notas e acordes, por exemplo).
\end{itemize}

Para o primeiro item, se prestam comumente os procedimentos de busca em tabelas com formas
de ondas em alta resolução. O procedimento é barato e, como vimos nos fundamentos,
não acrescentam ruídos relevantes ao sinal.

O segundo item presta-se à discretização do espaço musical. Unidades como batidas e notas
tornam mais eficiente a comunicação pois a quantidade
de estruturas sugeridas é maior~\cite{Roederer} no discreto doq eu no contínuo. Roederer
aponta que as próprias notas dos instrumentos musicais são um reflexo de que é mais eficiente
o uso do discreto do que do contínuo.

Desta forma, se mostram úteis para fazer sequencias de unidades, concatená-las. Quando as unidades
são notas, as sequencias sustapostas no tempo de unidades são melodias ou linhas melódicas. As
sequencias sobrepostas no tempo são comumente pensados como acordes, mas podem ser tidos simplesmente
como sobreposições circunstanciais de duas linhas melódicas.

As duas construções básicas escolhidas, baseadas na dicotomia melodia/harmonia, são
as funções fazSequencia e fazAcorde. Vale notar elas são absolutamente equivalentes e dispensáveis
em uma análise puramente conceitual. Os procedimentos de mixagem e concatenação são
plenamente capazes de realizar o que estas funções realizam. Aliás, as funções nada
mais são do que usos típicos e quase caricatos destes procedimentos: no fazAcorde a mixagem
sobrepõe no tempo todas as unidades, no fazSequencia as unidades são todas juntapostas no tempo.

 A função
lookup já foi apresentada na sessão anterior, quando tratado de Python já visando o áudio.
O lookupcruz é explicado a seguir.

[codigo do lookupcruz e explicacao]

As sequencias de notas e os acordes foram resolvidos com 2 outras funcoes simples,
que utilizam lookup e lookupcruz.

[codigo do fazSequencia e explicacao]

[codigo do fazAcorde e explicacao]

A função somador eh para automatizar a soma amostra a amostra e completar com zeros a
sequencia menor.

[codigo do somador, explicacao e figura de uma sequencia maior/menor que a outra]

Depois disso é usufruir com estruturas criadas. Por exemplo: depois de construidas
algumas tabelas para serem usadas como diferentes timbres, pode-se criar as
escalas completamente simetricas assim:

\code{Escalas Simétricas na grade dos 12 semintos e no âmbito da oitava}{python_snippets/escalas_simetricas.py}

% escala\_1=range(12) # cromática ascendente sem inclusão da oitava
% escala\_2=range(0,12,2) # tons inteiros
% escala\_3=range(0,12,3) # diminutão
% escala\_4=range(0,12,4) # terças maiores
% escala\_6=range(0,12,6) # trítonos
% \end{python}
% \input{py}

Sendo cada unidade um semitom. Assim as triades maiores e menores
são especificadas com simplicidade. Note que aos acordes \emph{diminutão} e
\emph{aumentado} correspodem as mesmas notas das escalas simetricas de terças menores
e maiores respectivamente:

\code{Acordes anotados como listas em python}{python_snippets/acordes.py}


E series podem ser anotadas junto a variacoes:

\code{Series diversas}{python_snippets/series.py}

Agora eh sintetizar e mixar. A sintese de sequencias e acordes sao feitos tipicamente
atraves de comandos desta forma:

\code{Sintetizando sequências e acordes}{python_snippets/seqs_acordes.py}

Aqui uns procedimentos basicos utilizados nas montagens:

\code{Amplificação e mixagem}{python_snippets/amp_mix.py}

Gerando uma serie de acordes:

\code{Acordes periódicos}{python_snippets/acordes_periodicos.py}





\subsubsection{Finite Groups in Granular and Unit Synthesis (FIGGUS)}

Desenvolvimento iniciado em 2006 com o físico-matemático Prof. Adolfo Maia Junior para
tratar de simetrias na música, com vistas à composição musical atraves
de métodos matemáticos. Mais especificamente, a proposta resultou em
um programa voltado para o uso de Grupos Algébricos para síntese
granular e síntese de estruturas musicais. O nome dado
foi FIGGUS, sigla de FInite Groups in Granular and Unit Synthesis.

A Síntese Granular é uma área bem estabelecida tanto na acústica quanto
na Computação Musical. O Físico Gabor, quantum sonoro, phonon, etc..
[achar artigos feitos com adolfo]

Na atual reescrita, embora ainda bastante atras do FIGGUS original,
mesmo sem contar a interface grafica, o codigo roda inteiro em python
com as biblitecas imbutidas por padrão. Isso permite com que o FIGGUS
sintetize todo um EP usando somente os comandos:

[[ARRUMAR A PARADA PARA O CODIGO]]
git clone
python setup.py install
python RUNME


  \subsection{Música em Tempo Real: Livecoding e ABeatTracker (ABT)}

  [Apresentar Livecoding.]

  Apenas parte dos conhecedores do assunto reconhecem
  o ABT como livecoding. Alguns
  consideram a linguagem por macros mais especializadas e poderosas como
  sendo um tipo de interface escrita, nao uma linguagem de programação
  propriamente dita.

      \subsubsection{ABeatTracker}


  Manual ABT

  Linkar com ABD

      \subsubsection{Audioexperiments, Estudiolivre e Foobarbaz}

      Audioexperiments. Estúdio livre. LM





  \subsection{Música na Matéria: EKP e AHT}
      \subsubsection{Emotional Kernel Panic (EKP)}

Em 2008, colaborando intensamente com o CDTL
(Centro de Desenvolvimento de Tecnologias Livres)\footnote{foi uma associação civil formada e desmembrada em 2008 e sediada em Recife, PE}
foi lançada a ideia de utilizar o estado do sistema operacional - especialmente o kernel linux - para
geração de sons. Surge o Emotional Kernel Panic (EKP). Desde o inicio, foram definidos três finalidades
para esta exploração do SO:

\begin{itemize}
    \item Didáticos
    \item Artísticos
    \item Monitoramento do SO
\end{itemize}

http://trac.assembla.com/audioexperiments/browser/ekp-base


      \subsubsection{AirHackTable}



  \section{Web}

  Difusão de informação com ênfase na facilitação
  da apropriação de tecnologias e de instancias políticas.

      \subsection{Tecnologias sociais de alta demanda: Sitios, Conteúdos e Articulação}

      \subsubsection{Sítios}

      FDDCA

      Ferramenta de comunicação

      (Cadastro dos pontos?)

      AA, SOS, Catalogo de Ideias, etc

      Meu site pessoal


      \subsubsection{Conteúdos}

      Wiki?

      \subsubsection{Articulação}

      IRC, Emails

\subsection{Disponibilização e desenvolvimento conjunto: wikis, etherpads, AA, Trac, IRC ..}

\subsubsection{Wiki}

\subsubsection{Trac}

\subsubsection{Screencasts - Vimeo}

\subsubsection{AA}

\subsubsection{Audio Experiments (Æ)}

\subsubsection{IRC}

\subsubsection{Etherpads}

\subsubsection{Outras fontes}


\section{Materiais didáticos}

  \subsection{Tutoriais em texto e código: Filtros, Nyquist e plugins LADSPA}

Os vários materiais didáticos produzidos constam no apêndice
deste trabalho. Um único destes será exposto a seguir por sua
capacidade de agregar os conteúdos dos capítulos anterioes:
os tutoriais de filtros e amostragem.

\begin{itemize}
    \item {\bf Tutorial de python para áudio e som}

Este tutorial foi levado para Berlim no LAC 2007 e sofreu melhoras desde entao. Esta
primeira versao ficou resumida em forma de texto no EL\footnote{http://estudiolivre.org/python-e-som-tutorial}. Em 2010
a Associacao Python Brasil escolheu este trabalho, então já mais amadurecido, para ser apresentado no
FISL em Porto Alegre. Como consequencia, foi feita uma série de video-tutoriais bastante utilizados\footnote{http://estudiolivre.org/tiki-index.php?page=Video+Tutoriais}.
Este tutorial foi comentado em listas em que o autor não participa (e outras em que o autor participa).

    \item {\bf Tutoriais de filtros e amostragem via python}

Voltados para explicitar principios fundamentais de áudio, estes tutoriais
são baseados código Python e o equivalente em C. Pequenas explicações são
dadas com o intuito de orientar a exploração inteligente destes \emph{snippets}.

\emph{Teorema de Amostragem}: estes scripts visam a experimentacao inteligente com
o Teorema de Nyquist. (descricao)

\emph{Filtros}: alem da explicitação sobre as diferenças entre filtros FIR e IIR,
duas utilizações clássicas destes filtros estão implementadas: Wavelets (FIR) e Quad (IIR)

Estes códigos podem ser baixados no repositório SVN do AudioExperiments. E os textos estão
na wiki (nos digitais ou EL, recriar pois os CDTL foram apagados)

    \item {\bf Tutorial de plugins lv2}

Dadas as dificuldades que o desenvolvimento dos \emph{plugins} de áudio apresenta,
desenvolvi um tutorial passo a passo com plugins que rodam em todas as etapas.
Ele é baseado em uma interface C++ para este padrão de plugin que eh implementado
em C. Os códigos e os textos estão todos em repositório.

    \item {\bf Microtutoriais Django ~\cite{dmicrotuts}}

Estes 'microtutoriais' são baseados nos conceitos de \emph{scripts mínimos} e
\emph{alterações puntuais}. O primeiro conjunto de microtutoriais é dedicado
a reconstruir o tutorial oficial do django de forma condensada e não prolixa.
O segundo destes conjuntos é dedicado a instrumentalizar de fato o leitor com
o entendimento do funcionamento dos princípios fundamentais deste framework.

    \item {\bf Philosometrics}

Embora este não seja um trabalho didático propriamente dito, ele tem este intuito
no cerne de sua concepção e surgimento. Em decorrência dele, surgiu o  Musimetrics,
o Cinemetrics e o Literametrics. Além disso, ele é um belo exemplo da
utilização das ciências duras para a análise de ciências humanas e foi acolhido
como tal em alguns momentos.

    \item {\bf Carta mídias livres}

Texto criado em decorrência da participação da comissão de seleção no
'Prêmio Mídias Livres', a convite do Ministério da Cultura por 'notório saber'.
Esta carta é um documento único no seu conteúdo, deixando às claras
o conceito de Mídias Livres como mídias não aprisionadas pelo conceito
de propriedade, ou seja, que priorizam a sua livre circulação e a possibilidade
de geração de materiais derivados, assim como sua geração aberta ao colaborativo e comunitário.

    \item {\bf Textos de cunho sociológico, transformador}

Produção mais numerosa que as anteriores, se caracteriza por métodos não convencionais
de abordagem dos assuntos e de escrita. Em especial utiliza-se pseudônimos para
auxiliar a despersonificação, gerando textos menos presos à satisfação da auto-imagem, dente
outras qualidades. A utiliação de psudônimos é um costume muito apreciado em diversos meios,
e as pesquisas tem confirmado as vantagens que a prática apresenta e confirma\footnote{http://disqus.com/research/pseudonyms/}.

O autor destes textos se dá ao direito de não revelar seus psudônimos - embora muitos deles
sejam publicamente conhecidos - para conservar as consequências desta prática na
forma mais pura. Como comprovante desta produção, deixamos uma mensagem sobre a publicação
de textos em mídia impressa com autores internacionais,
confirmando a participação do autor desta dissertação, mas cujo
nome não consta na publicação.

!!!!!!!!!!!!!!!!!!!!!!!!!!!!!!!!!!!!!!!

de      fabi borges catadores@gmail.com por  riseup.net 
responder a     submidialogia@lists.riseup.net
para    submidialogia@lists.riseup.net
data    23 de agosto de 2011 12:26
assunto Re: [submidialogia] livro sub- publicação
lista de e-mails        <submidialogia.lists.riseup.net> Filtrar as mensagens dessa lista de e-mails
enviado por     lists.riseup.net
assinado por    riseup.net
cancelar inscrição      Cancelar a inscrição para essa lista de e-mails
        Importante principalmente porque você frequentemente lê mensagens com esse marcador.
ocultar detalhes 12:26 (6 minutos atrás)
entao, eu fui recebendo textos durante esse tempo,
alguns tao atrazados como dos sem satelites, mas muita gente mandou;

aqui os autores:

os internacionais nao sao muitos, o joni kempf (o que bebe ouro do hardware), o barbrook (futuros imaginarios),
Hamdy heda (da revolucao egipcia), o pedro soller (summerlab), maria llopis (pos porno),  talvez a bronac, nao entregou ainda.

dos brasileiros, renato fabri, ruiz, pasteur, morgana e caio, ju dornelles, mari marcassa, coletivo errorista, adriana velozo-drica, tiago pimentel, felipe fonseca, thiago novaes, lelex, felipe ribeiro(?), maira, verenilde,  bartolina silva, poro, vitoria amaro, fabib (eu),

entao, precisa publicar agora,
uma equipe para publicacao e,,, 

bjs
f

!!!!!!!!!!!!!!!!!!!!!!!!!!!!!!!!!!!!!!!

\end{itemize}

\subsection{Screencasts e outros materiais em video e em texto}

\begin{itemize}
    \item Python para áudio e música
	  Texto - palestras - videos

    \item Canal Macambira
No Macambira estão sendo produzidos materiais em screencasts sobre
diversas cenas de hackeamento.

    \begin{itemize}
	\item Live-Coding
	\item Raspagem de dados
    \end{itemize}
\end{itemize}