%% ------------------------------------------------------------------------- %%
\chapter{Desenvolvimentos e resultados} %Nome do capítulo.
\label{cap:intro} %Rótulo para futura referência ao capítulo. Em qualquer lugar da tese, você poderá citar este capítulo através de ~\ref{cap:introducao}. Você escolhe o argumento de \label e pode ser qualquer coisa (Ex: \label{Procedimento_Experimental})

\section{Web}

Difusão de informação com ênfase na facilitação
da apropriação de tecnologias e de instancias políticas.

\subsection{Sítios}

FDDCA

Ferramenta de comunicação

Cadastro dos pontos

Meu site pessoal

\subsection{Conteúdos}

\subsubsection{textos}
Carta mídias livres

Artigos,

Tutoriais

\subsection{Articulação}

Em rede.


\section{Áudio}

\subsection{ABT}

Manual ABT

Linkar com ABD

\subsection{FDP}

Artigo FDP

\subsection{Experimentos abertos}

Audioexperiments. Estúdio livre.

\subsection{Tutoriais}

Screencasts. Textos.