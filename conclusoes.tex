%% ------------------------------------------------------------------------- %%
\chapter{Conclusões e trabalhos futuros} %Nome do capítulo.
\label{cap:conclusao}

No capítulo anterior está um sistema conciso
que relaciona elementos musicais ao som digital. \emph{Scripts} 
implementam estas relações, e em conjunto foram nomeados \massa\ (Música
e Áudio em Sequências e Séries Amostrais). 
A exposição didática destes desenvolvimentos no
capítulo anterior destina-se a facilitar a utilização
do arcabouço.

As possibilidades abertas por estes resultados envolvem a criação de interfaces de geração de ruídos e outros sons em alta fidelidade (\emph{hi-fi}), experimentos psicoacústicos e a utilização destes resultados para fins artísticos e didáticos. A incorporação de conhecimentos
em programação é bastante facilitada através de recursos audiovisuais, o que já realizamos por práticas de \emph{livecoding} e cursos focados em ferramentas especializadas, como o Puredata e o ChucK.
Está prevista a utilização destes resultados com
métodos de inteligência artificial para geração de materiais artísticos.

A disposição online destes conteúdos na forma de hipertexto junto aos códigos e exemplos sonoros, todos em licenças livres, facilita colaborações e geração de subprodutos em co-autoria, e com isso a expansão da \massa\ com novas implementações e desenvolvimentos das montagens musicais.
Explorações sistemáticas de parametrizações (dos tremolos, da ADSR, etc) em alta fidelidade tem utilidade artística e é possibilitada por este trabalho com controle amostral. Tal descrição analítica precisa, junto às implementações computacionais, não foi atingida anteriormente, como mostra o Apêndice~\ref{cap:trabalhosRelacionados} com uma visita aos trabalhos relacionados.

Este trabalho também teve resultados não previstos, como a formação de grupos
de interesse em torno da questão criativa aliada à computação.
Neste contexto, destaca-se o grupo
labMacambira.sf.net, que reúne colaboradores de todo o Brasil e alguns fora do país.
Este grupo
já apresentou contribuições relevantes em diferentes áreas
como Democracia Direta Digital, ferramentas de georreferenciamento e
atividades artísticas e educacionais, como cursos, workshops e apresentações artísticas. Vários destes resultados estão no Apêndice~\ref{cap:musicaExtra} e no acervo online, que ultrapassa 700 vídeos, documentações escritas, diversos software originais e contribuições em software externos utilizados no mundo todo, como o Firefox, Scilab, LibreOffice, GEM/Puredata, para citar somente alguns exemplos.\cite{siteLM,wikiLM,vimeoLM}

Há um aumento no número de pesquisas relacionadas à música em andamento no campus de São Carlos da USP, o que sugere facilidade para estabelecer parcerias.
As publicações acadêmicas efetivadas durante este mestrado também apontam para uma multidisciplinaridade,
tratando diretamente de questões humanas como artes, filosofia, humor e linguagem falada e escrita, através de artifícios lineares e estatísticos.\cite{FabbriSTAT,FabbriACL,FabbriComplenetVoz,FabbriComplenetTexto} Os desdobramentos estão alcançando redes sociais e teorias epistemológicas com base em pesquisas prévias, dos orientadores deste trabalho, com forte presença de redes complexas e processamento de linguagem natural. 
