%% ------------------------------------------------------------------------- %%
\chapter{Conclusões e trabalhos futuros} %Nome do capítulo.
\label{cap:conclusao}

Os resultados apresentados no capítulo anterior estão bastante de acordo
com os objetivos: conseguimos estabelecer um sistema conciso
que relaciona elementos musicais ao som digital. Dispomos também um conjunto
de scripts que implementam estas relações, ao qual chamamos MASSA (Música
e Áudio em Sequências e Séries Amostrais). Talvez em último lugar, mas igualmente
importante, realizamos uma exposição destes desenvovimentos de forma didática no
capítulo anterior.

Este trabalho também teve resultados não previstos, como a formação de grupos
de interesse em torno da questão criativa impulsionada por ferramentas computacionais,
em especial pelo código de programação. Neste contexto, se destaca o grupo
labMacambira.sf.net, que reúne colaboradores de todo o Brasil e alguns fora do país.
Este grupo tem vida própria com relação ao papel do orientado
deste trabalho, e já apresentou contribuições relevantes em diferentes áreas
como Democracia Direta Digital, ferramentas digitais de georeferenciamento e
atividades artísticas e educacionais, como cursos, workshops e apresentações artísticas. Vários destes resultados estão apresentados nos Apêndices~\ref{cap:mapa-sitio} e~\ref{cap:musicaExtra} além de todo acervo online criado, que ultrapassa 700 videos, diversos software originais e contribuições em software externos utilizados no mundo todo, como o Firefox, Scilab, LibreOffice, GEM/Puredata, para citar somente alguns exemplos.

As possibilidades previstas por estes resultados envolvem a criação de interfaces de geração de ruídos e outros sons em alta fidelidade (hi-fi), o auxílio na realização de experimentos psicoacústicos e, principalmente, a utilização destes resultados para fins artísticos e didáticos. A incorporação da linguagem de programação é bastante facilitada através de recursos audiovisuais, o que já realizamos por práticas de \emph{livecoding} e cursos focados em ferramentas especializadas, como o Puredata e o ChucK, e está em muito potencializada e confirmada através dos resultados descritos neste trabalho.

A disposição online destes conteúdos na forma de hipertexto junto aos códigos e exemplos sonoros, todos em licenças livres, não só facilita colaborações e geração de subprodutos em co-autoria, mas também que nossas contribuições tenham usos e se desdobrem de formas que ainda não concebemos.







