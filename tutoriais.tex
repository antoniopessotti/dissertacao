\chapter{Mais tutoriais e outros materiais didáticos} %Nome do capítulo.
\label{cap:tutsDidads}

\subsection{Tutoriais em texto e código}
\begin{itemize}
    \item Tutorial de python para áudio e som.

    \item Tutoriais de filtros e amostragem via python.

    \item Tutorial de plugins lv2

    \item Microtutoriais Django ~\cite{dmicrotuts}.

    \item Philosometrics

    \item Carta mídias livres

    \item Textos de cunho sociológico, transformador

\end{itemize}

\subsection{Screencasts}

\begin{itemize}
    \item Python para áudio e música

    \item Canal Macambira

    \begin{itemize}
	\item Live-Coding
	\item Raspagem de dados
    \end{itemize}
\end{itemize}



\subsection{Figusdevpack (FDP)}

Um ambiente de interação da comunidade de Python e Música 
para compartilhamento de códigos e excertos. Baseado principalmente
em documentação organizada sobre as práticas e as bibliotecas
existentes para python. Assim como scripts compartilhados que
fazem uso de objetos e módulos específicos. Este trabalho foi aceito na
maior conferência de áudio em linux, a Linux Audio Conference de 2008
(LAC2008) e está sendo reativado por mim em conjunto com Vilson Vieira
e outros desenvolvedores de áudio. Este projeto está em desenvolvimento
do site do Estúdio Livre [estudio livre] com repositório no sourceforge [source force].

As principais fontes sobre estre trabalho é a página de desenvolvimento da ideia
que está em constante mudança ~\cite{http://estudiolivre.org/tiki-index.php?page=fdp&highlight=fdp fdpel}
e, o artigo que foi aceito no LAC de 2008 ~\cite{http://www.estudiolivre.org/el-gallery_view.php?arquivoId=8221 fdplac2008}
e o repositório ~\cite{http://sourceforge.net/projects/fdpack/develop fdpsf}.