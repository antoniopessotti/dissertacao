\begin{resumo}

Nesta dissertacao, expomos os fundamentos do manuseio do som digitalizado através do software em código aberto.
Após uma brevíssima contextualização sobre o status que o código e o áudio digital possuem na cultura atual, procuramos abordar como um todo o fenômeno
físico (e psicofísico) do som, a sua representação digital e os procedimentos clássicos de manuseio. As linguagens de programação C/C++ e Python
são então abordadas, já com vistas a esta tarefa lidar com o áudio digital. Ainda no campo do software,
os sistemas operacionais são então perspassados e o ambiente de desenvolvimento Unix é descrito de um ponto
de vista bem resumido e ferramental. Um capítulo especial eh dedicado às linguagens dedicadas ao áudio e à música.
Os resultados antingidos com os fundamentos aqui expostos são então sistematizados com ênfase nas ferramentas desenvolvidas.
Apresentamos também uma série de repercussões destes desenvolvimentos.
Por fim, como um produto adicional relacionado a esta dissertação, é delineado o mapa de exposição e comunicação no qual
os desenvolvimentos deste trabalho estão agregados. Esta estrutura é dedicada a manter as contribuições
de forma continuada e sistematizada através de webpages, wikis, repositórios, um canal IRC, aplicativos de acompanhamento de projetos
e uma das tecnologias desenvolvidas neste trabalho: A Autogestão Algorítmica.


$\phantom{linha em branco}$\\
Palavras-chave: áudio, programação, acústica, código aberto, cultura digital, cultura hacker

\end{resumo}

\begin{abstract}

tradução de:

Nesta dissertacao, expomos os fundamentos do manuseio do som digitalizado através do software em código aberto.
Após uma brevíssima contextualização sobre o status que o código e o áudio digital possuem na cultura atual, procuramos abordar como um todo o fenômeno
físico (e psicofísico) do som, a sua representação digital e os procedimentos clássicos de manuseio. As linguagens de programação C/C++ e Python
são então abordadas, já com vistas a esta tarefa lidar com o áudio digital. Ainda no campo do software,
os sistemas operacionais são então perspassados e o ambiente de desenvolvimento Unix é descrito de um ponto
de vista bem resumido e ferramental. Um capítulo especial eh dedicado às linguagens dedicadas ao áudio e à música.
Os resultados antingidos com os fundamentos aqui expostos são então sistematizados com ênfase nas ferramentas desenvolvidas.
Apresentamos também uma série de repercussões destes desenvolvimentos.
Por fim, como um produto adicional relacionado a esta dissertação, é delineado o mapa de exposição e comunicação no qual
os desenvolvimentos deste trabalho estão agregados. Esta estrutura é dedicada a manter as contribuições
de forma continuada e sistematizada através de webpages, wikis, repositórios, um canal IRC, aplicativos de acompanhamento de projetos
e uma das tecnologias desenvolvidas neste trabalho: A Autogestão Algorítmica.


$\phantom{linha em branco}$\\
Keywords: audio, programming, acoustics, open source, digital culture, hacker culture

\end{abstract}