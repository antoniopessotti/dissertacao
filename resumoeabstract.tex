\begin{resumo}

A representação dos elementos básicos da música - tais como notas musicais, ornamentos e estruturas intervalares - em termos do som discretizado é bastante utilizada em softwares e rotinas para criação musical e tratamento sonoro. Não há, entretanto, um sistema unificado e conciso que relacione matematicamente estes elementos às sequências de amostras no tempo. Nesta dissertação, implementamos um sistema computacional ao qual chamamos MASSA (Música e Áudio em Sequências e Séries Amostrais) em que cada elemento musical é descrito por equações que resultam diretamente nas sequências temporais do som em sua representação discretizada. O elemento fundamental, a nota musical básica com duração, volume, altura e timbre, é relacionado quantitativamente às características do sinal digital. As variações internas das notas, como trêmolos, vibratos e flutuações espectrais, também são contempladas, o que permite sintetizar notas e sonoridades com inspiração nos instrumentos musicais reais além de sonoridades novas. A partir dessa representação das notas, é possível gerar estruturas musicais através de recursos consagrados, como a métrica rítmica, os intervalos de altura e os ciclos. As equações deram origem a uma implementação na linguagem Python, com \emph{scripts} que resultam em exemplos sonoros simples e imagens como espectrogramas, variações de amplitude e princípios organizacionais da música. A eficácia do sistema foi comprovada com a síntese de pequenas peças usando notas básicas, notas incrementadas e notas em música. É possível, também, sintetizar álbuns inteiros através de colagens de scripts e parametrização especificada pelo usuário. Com o paradigma de implementação em código aberto, o sistema pode ser expandido em processos de co-autoria e usado livremente por músicos e outros interessados. De fato, o sistema já vem sendo empregado por usuários externos para fins diversos, que incluem a produção de músicas, apresentações artísticas, experimentos psicoacústicos e difusão da linguagem computacional explorando o apelo lúdico do audiovisual.


$\phantom{linha em branco}$\\
Palavras-chave: som, acústica, áudio, código aberto, cultura digital, cultura hacker, programação

\end{resumo}

\begin{abstract}

tradução de:
Nesta dissertação, expomos os fundamentos do manuseio do som digitalizado através de recursos em código aberto como
uma forma de manifestação cultural.
Após uma brevíssima contextualização sobre o status que o código e o áudio digital possuem na cultura atual, procuramos abordar como um todo o fenômeno
físico (e psicofísico) do som, a sua representação digital e os procedimentos clássicos de manuseio. As linguagens de programação C/C++ e Python
são então abordadas já com vistas ã tarefa de lidar com o áudio digital. Um capítulo especial é dedicado às linguagens dedicadas ao áudio e à música. Ainda no campo do software,
o ambiente de desenvolvimento Unix é descrito de um ponto
de vista bem resumido e ferramental, pois é a base para todos os desenvolvimentos expostos nesta dissertação.
Os resultados antingidos com os fundamentos expostos são então sistematizados com ênfase nas ferramentas desenvolvidas.
Apresentamos também uma série de repercussões destes desenvolvimentos, em especial uma frente de ação criada: o LabMacambira.sf.net.
Por fim, como um produto adicional relacionado a esta dissertação, é delineado o mapa de exposição e comunicação no qual
os desenvolvimentos deste trabalho estão agregados. Esta estrutura é dedicada a manter as contribuições
de forma continuada e sistematizada através de webpages, wikis, repositórios, um canal IRC, aplicativos de acompanhamento de projetos
e uma das tecnologias desenvolvidas neste trabalho: a Autoregulação Algorítmica.


$\phantom{linha em branco}$\\
Keywords: audio, programming, acoustics, open source, digital culture, hacker culture

\end{abstract}
