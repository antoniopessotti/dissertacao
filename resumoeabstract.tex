\begin{resumo}

O código é repetidamente destacado como o bem mais valioso produzido em nosso tempo.
O código aberto - e sua abordagem politico-filosófica chamada de software livre -
se propõe a constituir um repertório tecnológico da humanidade, acessível por
qualquer ser humano. Esta proposta orienta o que é chamado de cultura digital,
um agregado de práticas e explorações recentes e advindas da popularização do digital.
É tido como o vetor de transformação de nossa época por excelência.

Após a contextualização, expomos os fundamentos teóricos de uma série de contribuições em áudio, música
e web, todas feitas em código aberto. Os resultados antingidos, assim como as repercussões
destes desenvolvimentos são então sistematizados por escrito. Por fim, como um produto adicional
deste trabalho, é delineado o mapa de um sítio no qual os desenvolvimentos desta tese
são agregados. Tal sítio é dedicado a manter as contribuições de forma
continuada e sistematizada.

$\phantom{linha em branco}$\\
Palavras-chave: código aberto, cultura digital, áudio, música, internet.

\end{resumo}

\begin{abstract}

tradução de:

O software livre se propõe a constituir um repertório tecnológico da humanidade. O código
é destacado como o bem mais valioso que é produzido em nosso tempo.

A cultura digital é o vetor de transformação de nossa época por excelência e seu representante
maior é o software livre.

Dada a contextualização, esmeramos as contribuições em código aberto de áudio e música
que foi desenvolvida, assim como alguns desenvolvimentos web e suas repercussões.

$\phantom{linha em branco}$\\
Keywords: open source, digital culture, audio, music, internet.

\end{abstract}