\begin{resumo}

Nesta dissertacao, expomos os fundamentos do manuseio do som digitalizado através de recursos em código aberto.
Após uma brevíssima contextualização sobre o status que o código e o áudio digital possuem na cultura atual, procuramos abordar como um todo o fenômeno
físico (e psicofísico) do som, a sua representação digital e os procedimentos clássicos de manuseio. As linguagens de programação C/C++ e Python
são então abordadas já com vistas ã tarefa de lidar com o áudio digital. Um capítulo especial é dedicado às linguagens dedicadas ao áudio e à música. Ainda no campo do software,
o ambiente de desenvolvimento Unix é descrito de um ponto
de vista bem resumido e ferramental, pois é a base para todos os desenvolvimentos expostos nesta dissertação.
Os resultados antingidos com os fundamentos expostos são então sistematizados com ênfase nas ferramentas desenvolvidas.
Apresentamos também uma série de repercussões destes desenvolvimentos, em especial uma frente de ação criada: o Lab Macambira.
Por fim, como um produto adicional relacionado a esta dissertação, é delineado o mapa de exposição e comunicação no qual
os desenvolvimentos deste trabalho estão agregados. Esta estrutura é dedicada a manter as contribuições
de forma continuada e sistematizada através de webpages, wikis, repositórios, um canal IRC, aplicativos de acompanhamento de projetos
e uma das tecnologias desenvolvidas neste trabalho: a Autogestão Algorítmica.


$\phantom{linha em branco}$\\
Palavras-chave: acústica, áudio, código aberto, cultura digital, cultura hacker, programação, som

\end{resumo}

\begin{abstract}

tradução de:

Nesta dissertacao, expomos os fundamentos do manuseio do som digitalizado através de recursos em código aberto.
Após uma brevíssima contextualização sobre o status que o código e o áudio digital possuem na cultura atual, procuramos abordar como um todo o fenômeno
físico (e psicofísico) do som, a sua representação digital e os procedimentos clássicos de manuseio. As linguagens de programação C/C++ e Python
são então abordadas já com vistas ã tarefa de lidar com o áudio digital. Um capítulo especial é dedicado às linguagens dedicadas ao áudio e à música. Ainda no campo do software,
o ambiente de desenvolvimento Unix é descrito de um ponto
de vista bem resumido e ferramental, pois é a base para todos os desenvolvimentos expostos nesta dissertação.
Os resultados antingidos com os fundamentos expostos são então sistematizados com ênfase nas ferramentas desenvolvidas.
Apresentamos também uma série de repercussões destes desenvolvimentos, em especial uma frente de ação criada: o Lab Macambira.
Por fim, como um produto adicional relacionado a esta dissertação, é delineado o mapa de exposição e comunicação no qual
os desenvolvimentos deste trabalho estão agregados. Esta estrutura é dedicada a manter as contribuições
de forma continuada e sistematizada através de webpages, wikis, repositórios, um canal IRC, aplicativos de acompanhamento de projetos
e uma das tecnologias desenvolvidas neste trabalho: a Autogestão Algorítmica.


$\phantom{linha em branco}$\\
Keywords: audio, programming, acoustics, open source, digital culture, hacker culture

\end{abstract}