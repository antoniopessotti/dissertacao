\chapter{Trabalhos relacionados e caracterização das contribuições deste trabalho}
\label{cap:trabalhosRelacionados}

Os trabalhos relacionados a esta dissertação são numerosos. Dentre as causas disso, pode-se apontar:

\begin{itemize}
    \item Natureza interdisciplinar entre música, computação e física.
    \item Há um interesse generalizado em música por parte das pessoas que compõem a sociedade.
    \item A programação de computadores está se difundindo notavelmente.
    \item As rotinas descritas neste trabalho são essenciais para boa parte dos \emph{software} voltados para áudio e música.
\end{itemize}

Embora as rotinas, apresentadas nesta dissertação, estejam presentes em diversas implementações livres e proprietárias, suas descrições precisas se encontram somente em código computacional. A maior contribuição desta dissertação é exatamente a descrição analítica das qualidades básicas que compõem elementos musicais no áudio digital. A apresentação didática dos fenômenos envolvidos também não foi encontrada na literatura visitada, o que, junto com as implementações em código Python destas relações e de peças musicais que as exemplifiquem, forma uma contribuição simples e convidativa embora inédita e multidisciplinar. No início da escrita desta dissertação, a caixa de ferramentas não estava prevista, ela foi fruto das equações e descrições precisas, o que tornou imediata a escrita dos scripts que compõem a \emph{toolbox} \massa.

Este capítulo é dedicado aos trabalhos similares ou relacionados. Os livros mais próximos são descritos. Na sequência, são apontadas as implementações computacionais proprietárias e livres. Por fim, esta dissertação é posicionada com relação aos trabalhos relacionados.

\section{Livros}\label{subsec:livros}

\begin{enumerate}
    \item \emph{Music For Geeks And Nerds: learn more about music with Python and a little bit of math}
        \begin{itemize}
            \item {\bf Descrição:} com exemplos em código computacional e sonoros, este excelente livro de Pedro Kroeger aborda conceitos de notas, afinações, especificação da nota Midi e conversão entre nomenclaturas latina (dó-ré-mi) e anglo-saxã (C-D-E). Lida também com operações musicais fundamentais como transposição, inversão e afins, combinações randômicas, por Fibonacci. Explora estas organizações tanto para acordes quanto para combinações horizontais (melódicas). Apresenta o básico sobre a constituição dos sons, batimentos, série harmônica, e um aprofundamento sobre as afinações. Por fim, aponta os recursos de ampliação temporal e de tessitura (alturas) de um dado conjunto de notas. Com isso, aponta o dipolo repetição/variação. Faz vínculos de apreensão de estruturas com peças de Josquin des Prez, Bach, Rachmaninoff e Steve Reich.
            \item {\bf Aspecto complementar:} as formalizações de operações dentro da notação tradicional são preciosos adendos às questões naturais abordadas na presente dissertação. A notação em si capta aspectos estruturais do sistema tonal e de 12 notas. Além disso, com a notação abre-se uma ponte com as tradições musicais eruditas.
            \item {\bf Aspecto diferencial:} o livro não desenvolve descrição precisa de aspectos musicais do som em si e não há foco em relacionar qualidades psicofísicas aos elementos musicais.
        \end{itemize}
    \item  \emph{The Theory and Technique of Electronic Music}
        \begin{itemize}
            \item {\bf Descrição:} um livro de Miller Puckette, de reconhecida complexidade, se define, nas palavras de Max Matheus (Prefácio) "The Theory and Technique of Electronic Music is a uniquely complete source of information for the computer synthesis of rich and interesting musical timbres". O livro começa com medições do som e controle de parâmetros, cai em síntese, modulações, métodos espectrais, atrasos e reverberações e termina com filtros.
            \item {\bf Aspecto complementar:} o livro apresenta diversos procedimentos valiosos para síntese, tratamento e análise. O texto busca ser de computação musical em geral e todo o texto é acompanhado de exemplos em Puredata, que é uma excelente linguagem de programação por patches voltada para audiovisual. Puredata é a linguagem de programação mais difundida na música acadêmica e tecnológica em geral.
            \item {\bf Aspecto diferencial:} salvo raras exceções, o livro não apresenta uma descrição analítica das amostras sonoras com relação aos procedimentos, assim, não relaciona de forma precisa as qualidades físicas do som. Tampouco se aprofunda em aspectos formais da teoria musical tradicional.
            \item {\bf Contribuições diretas:} na página 92, há uma solução para a o \emph{fade-in} e o \emph{fade-out} que, se feitos em progressão geométrica, demora a cair ao inaudível. A curva "quártica" atinge o zero e se é bastante próxima da progressão exponencial, especialmente nas intensidades maiores: $a_n = \left\{\left(\frac{n}{\Lambda-1}\right)^4\right\}_0^{\Lambda-1}$. Outra contribuição é a descrição prática do uso ideal de 1000 ou mais linhas de atrasos por segundo para simular a reverberação. Também deixa claro que há uma equalização na atenuação do som refletido, e que esta equalização tende ser mais atenuante nos agudos.
        \end{itemize}
    \item \emph{Real Sound Synthesis for Interactive Applications}
        \begin{itemize}
            \item {\bf Descrição:} livro do Perry Cook, de 2002, discorre sobre fundamentos de áudio digital e modelagem física. Possui preciosas dicas de modelagens de sons reais com características de instrumentos tradicionais e efeitos com origem nos instrumentos analógicos.
            \item {\bf Aspecto complementar:} implementa diversos instrumentos musicais e efeitos sonoros. Trata de uma biblioteca em C para síntese sonora que contempla boa parte das funcionalidades da \massa.
            \item {\bf Aspecto diferencial:} O trabalho escapa à descrição das amostras sonoras em si e não possui uma sistematização de elementos musicais básicos em termos das características sonoras.
            \item {\bf Contribuições diretas:}  a parte "1.3 Quantização" descreve de forma breve e suficiente o ruído de quantização e pode complementar à dissertação em "1.1 Som em áudio digital". O escrito sugere uma melhora na relação sinal/ruído de 6 decibels por bit utilizado na quantização de cada amostra do áudio PCM. O que indica diretamente uma relação sinal/ruído de 96dB para 16 bits/amostra (padrão de CD) e 48dB para 8 bits/amostra (comum em sistemas de voz).
        \end{itemize}
    \item \emph{Interação Tímbrica na Música Eletroacústica Mista}
        \begin{itemize}
            \item {\bf Descrição:} dissertação de Ignacio de Campos, possui diversas discussões cognitivas e musicais.
            \item {\bf Aspecto complementar:} desenvolvem os apontamentos da dissertação e do Apêndice~\ref{cap:musicaExtra} com relação à \emph{performance} musical.
            \item {\bf Aspecto diferencial:} o trabalho não lida de forma sistemática com as características do sinal digital de elementos musicais.
            \item {\bf Contribuições diretas:} a parte "2.3.2 A Sensação de Identidade Tímbrica" pode completar a exposição sobre timbre em "2.1.4 Timbre" com os apontamentos de envoltória, inarmonicidade, jitter e shimmer.
        \end{itemize}
    \item \emph{Music, Cognition, and Computerized Sound: An Introduction to Psychoacoustics}
        \begin{itemize}
            \item {\bf Descrição:} outro livro do Perry Cook, desta vez são artigos de vários autores, ele cuidou da edição. Um valioso apanhado pertinente para a psicofísica de elementos musicais no áudio digital, com textos focados em aspectos cognitivos e físicos do som além de eventuais descrições de procedimentos elétricos e digitais envolvidos. O foco é qualitativo embora com bastante embasamento quantitativo.
            \item {\bf Aspecto complementar:} o livro traz artigos de vários autores. As temáticas são pertinentes para a dissertação como cultura a respeito dos assuntos.
            \item {\bf Aspecto diferencial:} o trabalho não é uma descrição de elementos musicais com relação às características do som digital.
            \item {\bf Contribuições diretas:} talvez as partes mais interessantes para a dissertação sejam a subseção "23.10 Special Considerations in Psychoacoustic Research" e o capítulo "Appendix A: Suggested Lab Exercises".
        \end{itemize}
    \item \emph{Modelos Psicoacústicos de Dissonância para Eletrônica ao Vivo}
        \begin{itemize}
            \item {\bf Descrição:} tese de Alexandre Porres com considerações pertinentes sobre modelos psicoacústicos, dissonância e aspereza/rugosidade. Outros trabalhos, do mesmo autor, incluem um livro sobre computação musical em PD e apresentam, de forma acessível, procedimentos-chave de computação musical junto às implementações em Puredata.
            \item {\bf Aspecto complementar:} a tese foca em teorias psicoacústicas e em descrições minuciosas de rugosidade e dissonâncias que podem acrescentar bastante ao que foi apresentado na dissertação. Além disso, há um viés prático com a utilização do Puredata.
            \item {\bf Aspecto diferencial:} a trabalho não é uma descrição psicofísica de sequências amostrais relacionadas a elementos musicais.
        \end{itemize}
    \item \emph{5 livros do Julious O. Smith III}\cite{JOSFaust,JOSSpec,JOSFilt,JOSPhy,JOSFM}
        \begin{itemize}
            \item {\bf Descrição:} este autor possui diversos escritos, mais de 200 artigos publicados. Ao menos 5 livros de interesse para a dissertação: \emph{Mathematics of the Discrete Fourier Transform (DFT)} descreve a DFT minuciosamente, influindo FFT. O livro \emph{Introduction to Digital Filters (with audio applications)} trata de filtros de diversos tipos e descreve transformadas, técnicas de design de filtros e análises de frequência e por polos e zeros. Além de abordar filtros não convencionais, o livro termina com implementações em matlab, Faust e PD e possibilidades de confecção de plugins. O livro \emph{Physical Audio Signal Processing (for virtual musical instruments and audio effects)} trata de modelagem física. O livro \emph{Spectral Audio Signal Processing} adentra STFT em detalhes das janelas utilizadas e aplicações da STFT. Também aborda Wavelets de forma superficial porém consistente. O livro \emph{Audio Signal Processing in Faust} trata da linguagem Faust e suas facilidades para o processamento de áudio e música.
            \item {\bf Aspecto complementar:} a seção "Perceptual Aspects of Reverberation", do livro de filtros, pode fechar o assunto da espacialização junto à subseção "2.1.7 Localização espacial". As partes de modelagem de voz e de fundamentos físicos podem também servir de bons  complementos à dissertação. o livro sobre Faust complementa a dissertação por explicitar uma linguagem de domínio específico para áudio e música. O livro também contempla usos integrados com outros programas e linguagens via, por exemplo, o padrão de plugin LADSPA ou o protocolo OSC de comunicação entre programas para manipulação multimídia. 
            \item {\bf Aspecto diferencial:} estes trabalhos apresentam teorias fundamentais para o que se faz hoje em termos de programação para música, e não uma descrição de elementos musicais em termos das amostras digitais.
            \item {\bf Contribuições diretas:} a parte de espacialização tem como referencia forte o livro de modelagem física. A clareza sobre a natureza da reverberação e sobre suas características foi toda disparada por este livro.
        \end{itemize}
    \item \emph{Music: A Mathematical Offering}
        \begin{itemize}
            \item {\bf Descrição:} este livro, escrito por Dave Benson, se ocupa dos fundamentos matemáticos do fenômeno e tradições musicais.
            \item {\bf Aspecto complementar:} algumas afinações, escalas e explicações dos capítulos 5 e 6 (sobre afinações e escalas) podem facilitar algumas compreensões apontadas em subseção~\ref{subsec:afinacao}. O capítulo "9- Symmetry in Music" pode acrescentar nas estruturas cíclicas do capítulo 2.3 da dissertação, especialmente a parte de grupos, Cayley e órbitas.
            \item {\bf Aspecto diferencial:} o livro não trata de áudio digital. Além disso, o trabalho é uma coleção de modelagens matemáticas, não uma descrição do fenômeno sonoro-musical.
        \end{itemize}
    \item \emph{The Topos of Music: Geometric Logic of Concepts, Theory, and Performance}
        \begin{itemize}
            \item {\bf Descrição:} este impressionante livro de Guerino Mazzola possui quase 1500 páginas e explora diversos formalismos matemáticos para estruturas musicais. É um livro fascinante que atinge questões filosóficas.
            \item {\bf Aspecto complementar:} podem servir de ilustração ou aprofundamento: a subseções "6.5 Regular and Circular Forms", "22.4 Paradigmatic Groups" e "22.5 Pseudo-metrics on Orbits" e as seções "8 Symmetries and Morphisms" e "11 Orbits" fundamentam e ampliam a parte de estruturas cíclicas de 2.3. A subsubseção "7.2.2 Local Meters and Local Rhythms" e a seção "21 Metrics and Rythmics" podem aprofundar na parte de "2.3.4 Ritmo". A Subsubseção 9.4.2. pode dar uma contribuição na abordagem artística do código computacional. A subseções "9.4 Undestanding Fine and Other Arts" e "10.1 Paradigmata in Musicology, Linguistics, and Mathematics" e as seções "47 Unfolding Geometry and Logic in Time" e "48 Local and Global Strategies in Composition" podem fundamentar melhor a parte 2.3.7. Idioma Musical. Os tomos "VI Harmony" e "VII Counterpoint" podem aprofundar as partes "2.3.2 Rudimentos de harmonia" e "2.3.3 Contraponto". O tomo "XV Appendix: Sound" é bastante interessate, com algumas partes que lembram bastante a dissertação: "A Common Parameter Spaces", "A.1 Physical Spaces", "A.2.1 Onset and Duration", "A.2.2 Amplitude and Crescendo", "A.2.3 Frequency and Glussando" e "B Auditory Physiology and Psychology". No tomo "XVI Appendix: Mathematical Basics", interessa a parte "C.3 Groups" em que trata de homomorfismos, produtos direto, semidireto e 'Wreath', teoremas de Sylow, classificação, grupons affinos gerais e grupos de permutação, a subseção "E.3 Categories of Modules and Affine Transformations", "F.2 Spectra of Commutative Rings". O tomo "XVII Appendix: Tables" pode ter algumas afinações para constar em "2.3.1 Afinação, intervalos, escalas e acordes".
            \item {\bf Aspecto diferencial:} o livro não se ocupa da descrição de elementos musicais em termos do áudio digital.
        \end{itemize}
    \item \emph{Symmetry as a Compositional Determinant}
        \begin{itemize}
            \item {\bf Descrição:} livro Larry J. Salomon, de 1973 e revisado em 2002, é uma leitura de procedimentos composicionais tradicionais tendo como base comum a simetria.
            \item {\bf Aspecto complementar:} é um escrito bastante pertinente para toda a seção "2.3 Organização de notas em música", principalmente para a parte "2.3.6 Estruturas cíclicas" com as rotações, reflexões, translados e incidências tradicionais.
            \item {\bf Aspecto diferencial:} o trabalho não relaciona as características musicais com o som como fenômeno físico.
            \item {\bf Contribuições diretas:} foi um livro importante no amadurecimento das questões tratadas nesta dissertação por ser um livro focado em simetrias na música.
        \end{itemize}
    \item \emph{The CSound Book}
        \begin{itemize}
            \item {\bf Descrição:} editado por Richard Boulanger, 2000, possui contribuições de diversos autores. Além de abordar princípios da linguagem, seus funcionamentos internos e paradigmas musicais, é um apanhado de técnicas de síntese musical do som e estruturas de mais alto nível.
            \item {\bf Aspecto complementar:} apresenta a paradigmática linguagem de programação voltada para música: CSound. Junto a esta valiosa empreitada, consagrados recursos musicais eletrônicos são expostos.
            \item {\bf Aspecto diferencial:} é focado no CSound e não se propõe a ser um livro de fundamentos com descrição formal das características do som relacionado às estruturas musicais. 
        \end{itemize}
    \item \emph{The Audio Programming Book}
        \begin{itemize}
            \item {\bf Descrição:} de Richard Boulanger e Victor Lazzarini, 2011, um livro interessantíssimo que trata de programação em C/C++ voltado para áudio. Possui várias receitas tradicionais e fundamentais, todas implementadas em código C. Aborda CSound ao final. 
            \item {\bf Aspecto complementar:} aborda programação em C/C++, apresenta aspectos cruciais das implementações.
            \item {\bf Aspecto diferencial:} não é uma descrição do comportamento amostral do sinal sonoro com relação aos elementos musicais.
        \end{itemize}
    \item \emph{Computer Music Tutorial}
        \begin{itemize}
            \item {\bf Descrição:} de Curtis Roads, 1996, é já um clássico com algumas partes escritas por outros autores, como colaborações. 
            \item {\bf Aspecto complementar:} o capítulo "VII - Psychoacoustics" pode ser complementar à dissertação em algum detalhes e para aprofundamento.
            \item {\bf Aspecto diferencial:} Os textos do livro são de alto nível, sem preocupação central com o rigor, mas sim com apontamentos qualitativos e históricos das subáreas da música computacional. 
        \end{itemize}
    \item \emph{Music and Acoustics: From Instrument to Computer}
        \begin{itemize}
            \item {\bf Descrição:} um curto e preciso livro com os fundamentos acústicos, de modelagem de instrumentos musicais, de síntese e tratamento digitais.
            \item {\bf Aspecto complementar:} apresenta técnicas de modelagem física, além de aspectos musicais e computacionais.
            \item {\bf Aspecto diferencial:} não é uma descrição das amostras do sinal sonoro com relação às características musicais.
            \item {\bf Contribuição direta:} a proposta de modelar a reverberação como um ruido com decaimento exponencial foi importante para a elaboração da exposição sobre reverberação na subsubseção~\ref{subsubsec:reverb}
        \end{itemize}
\end{enumerate}

\section{Bibliotecas, linguagens e conjuntos de implementações computacionais voltados para música}\label{subsec:bibs}
{\bf Implementações fechadas.} As seguintes implementações fogem ao escopo do trabalho dada a inexistência de descrição amostral ou código que assegure uma implementação precisa e limpa. Como estas implementações são comerciais e o compromisso é com a satisfação do usuário, podem possuir pequenas adições de reverbs, delays e outros artifícios que, conhecidamente, fazem o resultado final soar mais agradável (chamado de \emph{'catchup'} no som ou na música). Observe que os plugins com funções adicionais às propostas pela classificação podem ser considerados adulterados. Exemplos desta adulteração são, em funções díspares do efeito especificado: aplicar um compressor de dinâmica para fazer o som ser ouvido como 'mais volumoso', aplicar reverb, dar ganho de equalização nos agudos ou nos graves, etc. 

\begin{enumerate}
    \item \emph{Plugins da Waves}: vários plugins que simulam equipamentos consagrados de estúdio analógico, além de montagem e configurações típicas de músicos e produtores famosos. Há plugins concebidos/desenhados por artistas/produtores com notória competência. Há uma série de plugins de um só parâmetro. Há uma linha de compressores, \emph{gates} e \emph{de-essers}. Uma linha especial de \emph{plugins} de \emph{limiters}. Uma extensa linha de excelentes \emph{plugins} de EQs. Uma linha de plugins de \emph{reverbs} e \emph{delays}. Outra de efeitos como Doppler, \emph{flanger}, \emph{morpher}, etc. Uma linha focada em \emph{plugins} de \emph{surround}. Uma linha bastante diferenciada de \emph{plugins} de remoção de ruído. Diversos \emph{plugins} para vocais. Alguns \emph{plugins} para alteração e modificação de altura. Outros analisadores. Por fim, alguns \emph{plugins} para criação de imagem estéreo e maximização da sensação do grave.
    \item \emph{GRM Tools} é uma coleção de plugins para transformação sonora que atinge, na prática cotidiana, resultados únicos e poderosos. É um conjunto de efeitos bastante considerado na música eletroacústica. Pode interessar à dissertação como exemplo de implementações modelo. 
    \item A \emph{Steinberg} possui uma série de programas considerados da melhor qualidade, incluindo os sequenciadores Cubase/Nuendo e o editor de áudio Wavelab. Há também diversos plugins VST (que é o padrão de plugin de áudio mais difundido e é da Steinberg) para síntese e tratamento sonoro. A Steinberg está também fornecendo bancos de sons. 
    \item \emph{Synths} (i.e. sintetizadores) mais conhecidos: Reason e ReCycle são produtos da Propellerheads com bastante ênfase em música de pista e simulação computacional de equipamentos e configurações de canais de áudio em hardware. O Tassman e outros intrumentos da Applied Acoustics Systems proporcionam sons de qualidade impressionante. Os instrumentos da Native Instrumens são consagrados, em especial o sintetizador ABSynth, o Traktor para VJing e o Maschine para fazer batidas; além destes, o sintetizador modular Reaktor e o pacote Komplete são também bastante conhecidos. 
\end{enumerate}

{\bf Implementações livres}. Podem ser consideradas documentações pois os códigos computacionais estão disponíveis e podem ser considerados especificações dos procedimentos envolvidos. 

\begin{enumerate}
    \item \emph{SndObj} é uma biblioteca em C usada também como um módulo Python. O manual está claro: "o SndObj é uma unidade de programação que pode gerar sinais com especificações de áudio e parâmetros de controle". A biblioteca tem anos de existência, foi escrita e é mantida por Victor Lazzarini com contribuições da FFTW e do Frank Barknecht. A documentação, incluindo sítio, manual e códigos, lida com especificações técnicas da biblioteca como um todo e com as entradas e saídas de cada objeto. É complementar com relação às implementações dos Apêndices. Não possui descrição analítica dos procedimentos.
    \item \emph{The MusicKit} é uma biblioteca em ObjC com mais de 10 anos de existência. Descreve-se como um sistema de \emph{software} orientado a objetos para criação musical. 
    \item \emph{Abjad} é uma biblioteca Python que se define como um "sistema de \emph{software} interativo para ajudar compositores a escreverem complexas peças de notação musical de forma incremental e iterativa". O Abjad utiliza o Lilypond para renderizar as estruturas em notação musical.
    \item \emph{music21}: esta biblioteca do MIT se descreve como um conjunto de ferramentas para musicologia auxiliada por computadores. Em grande parte, lida com a entrada e saída de elementos musicais em bancos de dados com obras dos compositores mais relevantes da música erudita. Trata principalmente de notação musical e análise de dados. Complementar à dissertação é esta própria ênfase em análise e notação musical. Diferencial que não trata de aspectos físicos do som ou sua representação amostrada no tempo, mas sim de aspectos que estão descritos pelo viés natural em "2.3 Notas em música".
    \item Embora ainda possa ser usado como uma biblioteca, o \emph{AthenaCL} tem já características de uso que fogem à linguagem Python, com terminal próprio e outras interfaces. O autor, Christopher Arizza, também co-responsável pelo Music21, o define com um "sistema que é uma ferramenta computacional orientada a objetos e em código aberto". O manual possui diversas das ferramentas, os detalhes de uso e implementações. É complementar quanto ao trabalho o aspecto de ferramenta computacional com linguagem própria e ferramentas encapsuladas à disposição. Diferencial é que não tem um tratamento analítico e não se atém às amostras em si. 
    \item \emph{PD e Max/MSP}: são linguagens de domínio específico (para música) que aproveitam a interface das linguagem de patches. As caixinhas cujo texto especifica o objeto e cujas entradas e saídas são ligadas por cordinhas são tanto uma interface amigável quanto uma interface gráfica para utilização do programa. Ambas as linguagens foram feitas em grande parte por Miller Puckette, que cuida atualmente somente da versão livre, que é o PD. O Max/MSP é mantido pela empresa Cycling 74. Os manuais do Max/MSP são ótimas e até históricas introduções qualitativas e práticas. O livro do Puckette, citado em na lista de livros acima, é um dos materiais excelentes que utilizam o PD. Complementar à dissertação há esta aproximação de um programa/sistema, com interface, e mais engessado do que uma linguagem de programação. Diferencial é que não há descrição analítica do que ocorre com as amostras. O código computacional, neste caso, é a documentação e Puckette explicita que o código do PD deva sempre ser completo se impresso em uma folha de papel.
    \item \emph{Faust, Chuck, Supercollider, Nyquist, Impromptu}: linguagens de domínio específico como p PD e o Max/MSP, voltadas para a síntese de música em tempo real, mas escritas normalmente em linhas, não em patches. Complementar ao trabalho é que são implementações computacionais excelentes para uso em tempo real. Diferencial é que há perda da desenvoltura para manipulações de áudio com controle amostral de precisão arbitrária. A descrição analítica também é inexistente. 
    \item \emph{CSound}: linguagem de programação voltada para música mais antiga que está em uso. As constantes produções acadêmicas em torno do CSound podem servir de descrições mais rigorosas do que as encontradas nas soluções anteriores. De qualquer forma, não há uma documentação clara e concisa sobre os procedimentos básicos como o capítulo 2 da dissertação. 
    \item \emph{tuneR}: um pacote em R para análise de música. A documentação é um bom manual dos objetos e as características que extraem. 
    \item \emph{Gwibber} e \emph{Vivace}: linguagens de programação voltadas para música (e video no caso do Vivace) utilizáveis via \emph{browser}.
\end{enumerate}


\section{Aprofundamento sobre esta dissertação com base nos trabalhos visitados nas subseções~\ref{subsec:livros} e~\ref{subsec:bibs}}

Há livros de técnicas de música eletrônica, como o do Puckette. Já materiais como o livro do Pedro Koeger e a biblioteca music21 não tratam das amostras sonoras em momento algum, mas sim da representação e manipulação de estruturas musicais, assunto abordado na seção~\ref{notasMusica} do ponto de vista natural. Há também trabalhos focados em modelagens matemáticas de técnicas e princípios musicais, cuja consideração permeia toda a dissertação e aprofundamentos são bem vindos. Estes trabalhos, quando muito, apresentam princípios de sinais discretos, mas não se ocupam da descrição do comportamento amostral. Os trabalhos de Cook, Smith III e algumas implementações possuem descrições de fenômenos vibratórios em corpos rígidos e outras modelagens físicas. Pode-se propor expansões para usos conjuntos com alguns dos trabalhos, visitados ou não. Vale assinalar: com base na experiência prévia dos autores, nos escassos artigos encontrados e nos trabalhos visitados em~\ref{subsec:livros} e~\ref{subsec:bibs}, a descrição analítica dos elementos musicais básicos em termos das amostras do áudio digital parece ser inédita. 
Como há também a implementação em código livre das relações descritas nesta dissertação na \emph{toolbox} \massa, abre-se um leque para experimentações científicas rigorosas assim como para usos artísticos com alta fidelidade e precisão arbitrária.\cite{MASSA} 
