\chapter{Trabalhos relacionados e caracterização das contribuições destre trabalho}
\label{cap:trabalhosRelacionados}

Os trabalhos relacionados a esta dissertação são numerosos. Dentre as causas disso, pode-se apontar:

\begin{itemize}
    \item Natureza interdisciplinar entre música, computação e física.
    \item Há um interesse generalizado em música por parte das pessoas que compõem a sociedade.
    \item A programação de computadores está se difundindo notavelmente.
    \item As rotinas descritas neste trabalho são essenciais para boa parte dos \emph{software} voltados para áudio e música.
\end{itemize}

Vale observar que, embora estas rotinas estejam presentes em diversas implementações livres e proprietárias, suas descrições precisas se encontram somente em código computacional. A maior contribuição desta dissertação é exatamente a descrição analítica das qualidades básicas que compõem elementos musicais no áudio digital. A apresentação didática dos fenômenos envolvidos também não foi encontrada na literatura visitada, o que, junto com as implementações em código Python destas relações e de peças musicais que as exemplifiquem, forma uma contribuição simples e convidativa embora inédita e multidisciplinar.

Este capítulo é dedicado aos trabalhos similares ou relacionados. Os livros mais próximos são descritos. Na sequência, são apontadas as implementações computacionais proprietárias e livres. Por fim, esta dissertação é posicionada com relação aos trabalhos relacionados.

\section{Trabalhos relacionados}

\subsection{Livros}

\subsection{Bibliotecas, linguagens e conjuntos de implementações computacionais voltados para música}

\subsection{Aprofundamento sobre esta dissertação com base nos trabalhos visitados}


