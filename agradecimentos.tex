

Agradeço ao Prof. Luciano da Fontoura Costa e ao Prof. Osvaldo Novais de Oliveira Junior pela oportunidade e pela orientação deste trabalho.

\vspace{4 mm}

Agradeço ao corpo de funcionários do IFSC, pela prestatividade e eficiência em todos os momentos que precisei.

\vspace{4 mm}

Agradeço à minha família, em especial à minha esposa Thaís Teixeira Fabbri e ao meu filho Antônio Anzoategui Fabbri.

\vspace{4 mm}

Agradeço ao meu irmão Ricardo Fabbri e ao amigo Vilson Vieira da Silva Junior pelas profícuas colaborações acadêmicas, artísticas e socialmente engajadas.


\vspace{4 mm}

Agradeço aos participantes do labMacambira.sf.net pelas colaborações diretas e indiretas nos software, apresentações e articulações desde junho de 2011. Em especial agradeco ao Daniel Penalva, Caleb Macarenhas, Chico Simões, Fábio Simões, Daniel Marostegan, Marcos Mendonça, Geraldo Magela, Patrícia Ferraz, Glerm Soares, Vanessa Ferreira, Sérgio Teixeira de Carvalho. Agradeço aos demais colaboradores das listas de email, IRC, AA e outros canais.


\vspace{4 mm}

Agradeço ao Cultura Viva e aos pontos de cultura pelo apoio fundamental a este e outros trabalhos. Agradeço em especial a estes Pontões e Cultura Digital e Ação Griô pelo suporte a este trabalho em ações regionais e nacionais: CDTL (PE), JuntaDados.org (BA), Nós Digitais (SP), Casa dos Meninos (SP), Nina Griô (SP), Pontão da Eco (RJ), Casa de Cultura Tainã (SP) e à Comissão Nacional dos Pontos de Cultura - CNPdC (GO).


\vspace{4 mm}

Agradeço às comunidades de cultura e software livre por todos os conhecimentos e tecnologias repassados e que compõem esta contribuição. 

%
%
%
%Em primeiro lugar, agradeço aos meus pais e demais familiares. Agradeço aos
%meus amigos pelo carinho e companhia. Aos professores por tanta atenção e compreensão.
%Aos colegas colecionadores de entendimentos por ter-nos construído até aqui.
%Agradeço a Felipe Machado pela visão, a Fabiana 'Goa' Sherine pelo entendimento,
%a Glerm Soares pela transcendência, a Fabianne Baveldi pela excelência, dentre tantos outros que não há páginas para escrever aqui. Agradeço ao profícuo Vilson Vieira, que tornou-se especialmente próximo
%das contribuições citadas neste trabalho, construindo sobre elas e com elas. Condizente com a fortíssima referência
%para que é para mim, tanto para os assuntos tratados nesta dissertação quanto para questões humanas, um especial agradecimento
%ao meu irmão Ricardo Fabbri que também se aproximou destas contribuições tanto cuidando de processamento de imagens e video quanto articulando e amadurecendo as ideias. Agradecimento
%forte ao LabMacambira.sf.net. Um profundo agradecimento
%aos meus orientadores de IC: Adolfo Maia Junior e Jônatas Manzolli. Agradeço a José Augusto Mannis
%pelos trabalhos de estudante, pelo acompanhamento e ensinos sobre manipulação de áudio e música como idioma.
%Agradeço fortemente ao Prof. Dr. Rafael Santos Mendes pela pesquisa em Wavelets de mestrado em engenharia elétrica
%praticamente completa na FEEC/Unicamp, titulo este que não defendendi por razões circunstanciais. Agradeço aos professores da FEEC pelos ensinos de
%primeiríssima qualidade em diversas disciplinas como processamento de sinais, teoria da informação, 
%circuitos elétricos, processos estocásticos
%e programação bio-inspirada. Sem sombra de dúvida, os maiores agradecimentos para os assíduos mentores
%deste trabalho: ao orientador-colaborador Prof. Luciano da Fontoura Costa e ao meu 
%orientator Prof. Osvaldo Novais de Oliveira Júnior. Ambos proporcionaram inestimável acompanhamento e notável paciência, dedicação e precisão em todos os momentos.
