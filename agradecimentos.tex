Em preimeiro lugar, agradeço aos meus pais e demais familiares. Agradeço aos
meus amigos pelo carinho e companhia. Aos professores por tanta atenção e compreensão.
Aos colegas colecionadores de entendimentos por ter-nos construído até aqui.
Agradeço a Felipe Machado pela visão, a Fabiana 'Goa' Sherine pelo entendimento,
a Glerm Soares pela transcendência, a Fabianne Baveldi pela excelência, a Daniel
Marostegan e Carneiro pelas lições e confianças. Agradeço ao profícuo Vilson Vieira, que tornou-se especialmente próximo
das contribuições citadas neste trabalho, construindo sobre elas e com elas. Especial agradecimento
ao meu irmao Ricardo Fabbri que também se aproximou desta contribuição cuidando de processamento de imagens e video. Fortíssima referência
para mim, tanto para os assuntos tratados nesta dissertação, quanto para questões humanas. Agradecimento
forte ao Lab Macambira, nominalmente estes assíduos: [Equipe Lab Macambira]. Um profundo agradecimento
aos meus orientadores de IC: Adolfo Maia Junior e Jônatas Manzolli. Agradeço a José Augusto Mannis
pelos trabalhos de estudante, pelo acompanhamento e ensinos sobre manipulação de áudio e música como idioma.
Agradeço fortemente ao Prof. Dr. Rafael Santos Mendes pela pesquisa em Wavelets de mestrado em engenharia elétrica
praticamente completa na FEEC/Unicamp que não defendendi por razões circunstanciais. Agradeço aos professores da FEEC pelos ensinos de
primeiríssima qualidade em diversas disciplinas como processamento de sinais, teoria da informação, 
circuitos elétricos, processos estocásticos
e programação bio-inspirada. Sem sombra de dúvida agradecimentos especiais para os assíduos mentores
deste trabalho: ao orientador-colaborador Prof. Luciano da Fontoura Costa, que, dentre outras qualidades,
mistura criatividade com clareza de pensamento. Finalmente, agradeço ao meu 
orientator Prof. Osvaldo Novais de Oliveira Júnior pelo inestimável acompanhamento
e notável paciência, dedicação e precisão em todos os momentos.