\chapter{Código Computacional das Peças Musicais}
\label{cap:codigoPecas}
Todas as peças a seguir se propõem a exemplificar as relações apresentadas no capítulo~\ref{cap:resultados} e são disponibilizadas online junto ao toolbox \massa.\cite{MASSA}


\subsection{Quadros sonoros}\label{ap:quadros}
Montagem musical 'quadros sonoros' para demonstração da mixagem pela soma direta de sequências amostrais. Resultam em 5 pequenas peças de sonoridades estáticas. Peça demonstrativa dos conceitos apresentados na seção~\ref{sec:notaDisc}.
\code{Quadros sonoros 1-5}{scripts/pecas2.1/quadrosSonoros.py}

\subsection{Reduced-fi}\label{ap:reduced}
Pequena peça musical para demonstração da concatenação de sequências amostrais como notas musicais. Resulta em uma pequena peça de 25 segundos. Peça demonstrativa dos conceitos apresentados na seção~\ref{sec:notaDisc} e disponibilizada online junto ao toolbox \massa.
\code{reduced-fi}{scripts/pecas2.1/reduced-fi-limpo.py}



\subsection{Transita para metro}\label{ap:transita}
Montagem musical 'transita para metro' para . Resultam em 1 pequenas peças com chirps. Peça demonstrativa dos conceitos apresentados na seção~\ref{sec:varInternas}.
\code{Transita para metro}{scripts/pecas2.2/transitaParaMetro.py}

\subsection{Vibra e treme}\label{ap:vibra}
Pequena peça musical para demonstração da . Resulta em uma pequena peça de XX segundos. Peça demonstrativa dos conceitos apresentados na seção~\ref{varInternas} e disponibilizada online junto ao toolbox \massa.
\code{Tremolos, vibratos e a frequência}{scripts/pecas2.2/vibraEtreme.py}





\subsection{Tremolos, vibratos e a frequência}\label{ap:tremolos}
Pequena peça musical para demonstração da . Resulta em uma pequena peça de XX segundos. Peça demonstrativa dos conceitos apresentados na seção~\ref{varInternas} e disponibilizada online junto ao toolbox \massa.
\code{Tremolos, vibratos e a frequência}{scripts/pecas2.2/TremolosVibratosEaFrequencia.py}



\subsection{Trenzinho de caipiras impulsivos}\label{ap:trenzinho}
Pequena peça musical para demonstração da . Resulta em uma pequena peça de XX segundos. Peça demonstrativa dos conceitos apresentados na seção~\ref{varInternas} e disponibilizada online junto ao toolbox \massa.
\code{Trenzinho de caipiras impulsivos}{scripts/pecas2.2/trenzinhoCaipiraImpulsivos.py}



\subsection{Ruidosa faixa}\label{ap:ruidosa}
Pequena peça musical para demonstração da . Resulta em uma pequena peça de XX segundos. Peça demonstrativa dos conceitos apresentados na seção~\ref{varInternas} e disponibilizada online junto ao toolbox \massa.
\code{Ruidosa faixa}{scripts/pecas2.2/ruidosaFaixa4.py}



\subsection{Bela Rugosi}\label{ap:bela}
Pequena peça musical para demonstração da . Resulta em uma pequena peça de XX segundos. Peça demonstrativa dos conceitos apresentados na seção~\ref{varInternas} e disponibilizada online junto ao toolbox \massa.
\code{Bela Rugosi}{scripts/pecas2.2/bellaRugosi.py}



\subsection{Chorus infantil}\label{ap:chorus}
Pequena peça musical para demonstração da . Resulta em uma pequena peça de XX segundos. Peça demonstrativa dos conceitos apresentados na seção~\ref{varInternas} e disponibilizada online junto ao toolbox \massa.
\code{Bela Rugosi}{scripts/pecas2.2/chorusInfantilb.py}



\subsection{ADa e SaRa}\label{ap:ada}
Pequena peça musical para demonstração da . Resulta em uma pequena peça de XX segundos. Peça demonstrativa dos conceitos apresentados na seção~\ref{varInternas} e disponibilizada online junto ao toolbox \massa.
\code{ADa e SaRa}{scripts/pecas2.2/ADa_e_SaRa.py}


%%%%%%%%%%%%%%%%%%%%%%%%%%%%%%%%%%%%%%
%%%% SECAO 2.3

\subsection{Intervalos entre alturas}\label{ap:intervalos}
Pequena peça musical para demonstração da . Resulta em uma pequena peça de XX segundos. Peça demonstrativa dos conceitos apresentados na seção~\ref{notasMusica} e disponibilizada online junto ao toolbox \massa.
\code{Intervalos entre alturas}{scripts/pecas2.3/intervalosEntreAlturas.py}


\subsection{Acorde cedo}\label{ap:acorde}
Pequena peça musical para demonstração da . Resulta em uma pequena peça de XX segundos. Peça demonstrativa dos conceitos apresentados na seção~\ref{notasMusica} e disponibilizada online junto ao toolbox \massa.
\code{Acorde cedo}{scripts/pecas2.3/acordeCedo.py}


\subsection{Conta ponto}\label{ap:conta}
Pequena peça musical para demonstração da . Resulta em uma pequena peça de XX segundos. Peça demonstrativa dos conceitos apresentados na seção~\ref{notasMusica} e disponibilizada online junto ao toolbox \massa.
\code{Cristais}{scripts/pecas2.3/contaPonto.py}



\subsection{Cristais}\label{ap:cristais}
Pequena peça musical para demonstração da . Resulta em uma pequena peça de XX segundos. Peça demonstrativa dos conceitos apresentados na seção~\ref{notasMusica} e disponibilizada online junto ao toolbox \massa.
\code{Cristais}{scripts/pecas2.3/cristais.py}


\subsection{Micro tom}\label{ap:micro}
Pequena peça musical para demonstração da . Resulta em uma pequena peça de XX segundos. Peça demonstrativa dos conceitos apresentados na seção~\ref{notasMusica} e disponibilizada online junto ao toolbox \massa.
\code{Cristais}{scripts/pecas2.3/microTom.py}



\subsection{Poli Hit Mia}\label{ap:poli}
Pequena peça musical para demonstração da . Resulta em uma pequena peça de XX segundos. Peça demonstrativa dos conceitos apresentados na seção~\ref{notasMusica} e disponibilizada online junto ao toolbox \massa.
\code{Cristais}{scripts/pecas2.3/poliHitMia.py}


\subsection{Dirracional}\label{ap:dirracional}
Pequena peça musical para demonstração da . Resulta em uma pequena peça de XX segundos. Peça demonstrativa dos conceitos apresentados na seção~\ref{notasMusica} e disponibilizada online junto ao toolbox \massa.
\code{Cristais}{scripts/pecas2.3/dirracional.py}


