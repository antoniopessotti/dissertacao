\chapter{Código Computacional das Peças Musicais}
\label{cap:codigoPecas}
Todas as peças a seguir foram feitas para exemplificar as relações apresentadas no capítulo~\ref{cap:resultados} e são disponibilizadas online junto ao toolbox \massa.\cite{MASSA}

\clearpage
\section{Peças referentes à seção~\ref{sec:notaDisc}}

\subsection{Quadros sonoros}\label{ap:quadros}
A montagem musical 'quadros sonoros' é decidada para demonstração da mixagem pela soma direta de sequências amostrais (subseção~\ref{subsec:basMus}). Esta rotina sintetiza 5 pequenas peças de sonoridades estáticas. Peça demonstrativa dos conceitos apresentados na seção~\ref{sec:notaDisc}. Houve relatos em listas de \emph{emails} de que estas sonoridades causaram alteração do estado de consciência. Em especial, o quadro 5 causou em diferentes pessoas o mesmo efeito: a sensação de escutar sons da mandimbula. 

\code{Quadros sonoros 1-5}{scripts/pecas2.1/quadrosSonoros.py}
\clearpage


\subsection{Reduced-fi}\label{ap:reduced}
Pequena peça musical para a concatenação de sequências amostrais como notas musicais (subseção~\ref{subsec:basMus}). Sintetiza uma pequena peça de 25 segundos em Python puro, i.e. sem utilizar bibliotecas externas como \emph{Numpy} ou \emph{Scikits/Audiolab}. Peça demonstrativa dos conceitos apresentados na seção~\ref{sec:notaDisc} e disponibilizada online junto ao \emph{toolbox} \massa.\cite{MASSA}
\code{reduced-fi}{scripts/pecas2.1/reduced-fi.py}


\clearpage
\section{Peças referentes à seção~\ref{varInternas}}

\subsection{Transita para metro}\label{ap:transita}
Sintetiza em 1 pequena peça de 49 segundos com varreduras de frequência, chamados \emph{chirps}. Também é proposta para explorações de transições de intensidade. Peça demonstrativa dos conceitos apresentados na subseção~\ref{subsec:vars} e disponibilizada junto à \massa.\cite{MASSA}
\code{Transita para metro}{scripts/pecas2.2/transitaParaMetro.py}

\clearpage
\subsection{Vibra e treme}\label{ap:vibra}
Para explorações de tremolos e vibratos como expostos na subseção~\ref{subsec:tvaf}, sintetiza 17 pequenas montagens de 8 a 24 segundos cada. Script disponibilizado online junto ao toolbox \massa.\cite{MASSA}
%\code{Vibra e treme}{scripts/pecas2.2/vibraEtreme.py}



\subsection{Tremolos, vibratos e a frequência}\label{ap:tremolos}
Pequena montagem musical para demonstração de vínculo entre os parâmetros dos tremolos e vibratos com a frequência fundamental central da nota. Resulta em 19 pequenas montagens de 4-32 segundos. Peça demonstrativa dos conceitos apresentados na subseção~\ref{subsec:mus2} e disponibilizada online junto ao toolbox \massa.\cite{MASSA}
%\code{Tremolos, vibratos e a frequência}{scripts/pecas2.2/TremolosVibratosEaFrequencia.py}



\subsection{Trenzinho de caipiras impulsivos}\label{ap:trenzinho}
Pequena peça musical para demonstração do deslocamento causado pela convolução com o impulso. Sintetiza 11 pequenas montages cada uma com duração de 4-240 segundos. Peça demonstrativa dos conceitos apresentados na subseção~\ref{subsec:filtros} e disponibilizada online junto ao toolbox \massa.
%\code{trenzinho de caipiras impulsivos}{scripts/pecas2.2/trenzinhoCaipiraImpulsivos.py}



\subsection{Ruidosa faixa}\label{ap:ruidosa}
Pequena peça musical de 240 segundo spara demonstração de filtragens diversas em ruídos e para reverberação, conceitos apresentados na subseção~\ref{subsec:filtros} e disponibilizada online junto ao toolbox \massa.\cite{MASSA}
\code{ruidosa faixa}{scripts/pecas2.2/ruidosaFaixa4.py}

\clearpage


\subsection{Bela Rugosi}\label{ap:bela}
Pequena peça musical de 96 segundos para demonstração da rugosidade causada por oscilações com frequencias $\approx 13-30 Hz$, conceitos apresentados na subseção~\ref{subsec:tvaf}. A peça é também disponibilizada online junto à toolbox \massa.\cite{MASSA}
\code{bela rugosi}{scripts/pecas2.2/bellaRugosi.py}

\clearpage


\subsection{Chorus infantil}\label{ap:chorus}
Script para demonstração do efeito \emph{chorus} (inspirado em coro de cantores) . Sintetiza 4 pequenas montagens de 8-32 segundos com os conceitos apresentados na subseção~\ref{subsec:mus2} e disponibilizada online junto ao toolbox \massa.\cite{MASSA}
%\code{chorus infantil}{scripts/pecas2.2/chorusInfantil.py}



\subsection{ADa e SaRa}\label{ap:ada}
Pequenas peças musicais para demonstração da envoltória ADSR. Sintetiza 7 peças de 5-17 segundos, cetradas nos conceitos apresentados na subseção~\ref{subsec:mus2} e disponibilizada online junto à \massa.\cite{MASSA}
%\code{ADa e SaRa}{scripts/pecas2.2/ADa_e_SaRa.py}


\clearpage
%%%%%%%%%%%%%%%%%%%%%%%%%%%%%%%%%%%%%%
%%%% SECAO 2.3

\section{Peças referentes à seção~\ref{notasMusica}}
\subsection{Intervalos entre alturas}\label{ap:intervalos}
Pequena peça musical de 45 segundos para exploração do sistema de intervalos, demonstrativa dos conceitos apresentados na subseção~\ref{subsec:intervalos}, e disponibilizada online junto ao toolbox \massa.
%\code{intervalos entre alturas}{scripts/pecas2.3/intervalosEntreAlturas.py}


\subsection{Cristais}\label{ap:cristais}
Pequena peça musical de 64 segundos para demonstração das escalas simétricas, conceitos apresentados na subseção~\ref{subsec:intervalos}, e disponibilizada online junto à \massa.\cite{MASSA}
\code{cristais}{scripts/pecas2.3/cristais.py}

\clearpage


\subsection{Micro tom}\label{ap:micro}
Pequena peça musical para demonstração da microtonalidade de quartos de tom e de sétimos de oitava, conceitos apresentados na subseção~\ref{subsec:intervalos}, e disponibilizada online junto à \massa.\cite{MASSA}
\code{micro tom}{scripts/pecas2.3/microTom.py}


\clearpage

\subsection{Acorde cedo}\label{ap:acorde}
Pequena peça musical de 40 segundos para demonstração dos encadeamentos tonais básicos, conceitos apresentados na subseção~\ref{subsec:harmonia}, e disponibilizada online como parte da \massa.\cite{MASSA}
%\code{acorde cedo}{scripts/pecas2.3/acordeCedo.py}


\subsection{Conta ponto}\label{ap:conta}
Pequena montagem musical de 17 segundos para demonstração de vozes conduzidas por regras simples para manterem independência, conceitos apresentados na subseção~\ref{subsec:contraponto}, e disponibilizada online junto à \massa.\cite{MASSA}
%\code{conta ponto}{scripts/pecas2.3/contaPonto.py}




\subsection{Poli Hit Mia}\label{ap:poli}
Pequena montagem musical de 99 segundos para demonstração da métrica musical e das diferentes contagens, conceitos apresentados na seção~\ref{subsec:ritmo}, e disponibilizada online junto à \massa.\cite{MASSA}
%\code{poli hit mia}{scripts/pecas2.3/poliHitMia.py}



\subsection{Dirracional}\label{ap:dirracional}
Pequena peça musical de 22 segundos para demonstração das estruturas direcionais e posicionalmentos dos clímax, conceitos apresentados na seção~\ref{notasMusica}, e disponibilizada online junto à \massa.\cite{MASSA}
\code{dirracional}{scripts/pecas2.3/dirracional.py}


