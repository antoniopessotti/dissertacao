\chapter{Código Computacional das Peças Musicais}
\label{cap:codigoPecas}
Todas as peças a seguir foram feitas para exemplificar as relações apresentadas no capítulo~\ref{cap:resultados} e são disponibilizadas online junto ao toolbox \massa.\cite{MASSA}


\subsection{Quadros sonoros}\label{ap:quadros}
Montagem musical 'quadros sonoros' para demonstração da mixagem pela soma direta de sequências amostrais. Esta rotina sintetiza 5 pequenas peças de sonoridades estáticas. Peça demonstrativa dos conceitos apresentados na seção~\ref{sec:notaDisc}.
\code{Quadros sonoros 1-5}{scripts/pecas2.1/quadrosSonoros.py}

\clearpage

\subsection{Reduced-fi}\label{ap:reduced}
Pequena peça musical para demonstração da concatenação de sequências amostrais como notas musicais. Sintetiza uma pequena peça de 25 segundos em Python puro, i.e. sem utilizar bibliotecas externas como Numpy ou Scikits/Audiolab. Peça demonstrativa dos conceitos apresentados na seção~\ref{sec:notaDisc} e disponibilizada online junto ao toolbox \massa.\cite{MASSA}
\code{reduced-fi}{scripts/pecas2.1/reduced-fi.py}


\clearpage

\subsection{Transita para metro}\label{ap:transita}
Sintetiza em 1 pequena peça de 49 segundos com varreduras de frequência, chamados chirps. Também é proposta para explorações de transições de intensidade. Peça demonstrativa dos conceitos apresentados na subseção~\ref{subsec:vars} e disponibilizada junto à \massa.\cite{MASSA}
\code{Transita para metro}{scripts/pecas2.2/transitaParaMetro.py}

\clearpage
\subsection{Vibra e treme}\label{ap:vibra}
Para explorações de tremolos e vibratos como expostos na subseção~\ref{subsec:tvaf}, sintetiza 17 pequenas montagens de 8 a 24 segundos cada. Script disponibilizado online junto ao toolbox \massa.\cite{MASSA}
\code{Vibra e treme}{scripts/pecas2.2/vibraEtreme.py}


\clearpage

\subsection{Tremolos, vibratos e a frequência}\label{ap:tremolos}
Pequena montagem musical para demonstração de vínculo entre os parâmetros dos tremolos e vibratos com a frequência fundamental central da nota. Resulta em 19 pequenas montagens de 4-32 segundos. Peça demonstrativa dos conceitos apresentados na subseção~\ref{subsec:mus2} e disponibilizada online junto ao toolbox \massa.\cite{MASSA}
\code{Tremolos, vibratos e a frequência}{scripts/pecas2.2/TremolosVibratosEaFrequencia.py}


\clearpage

\subsection{Trenzinho de caipiras impulsivos}\label{ap:trenzinho}
Pequena peça musical para demonstração do deslocamento causado pela convolução com o impulso. Sintetiza 11 pequenas montages cada uma com duração de 4-240 segundos. Peça demonstrativa dos conceitos apresentados na subseção~\ref{subsec:filtros} e disponibilizada online junto ao toolbox \massa.
\code{trenzinho de caipiras impulsivos}{scripts/pecas2.2/trenzinhoCaipiraImpulsivos.py}


\clearpage

\subsection{Ruidosa faixa}\label{ap:ruidosa}
Pequena peça musical de 240 segundo spara demonstração de filtragens diversas em ruídos e para reverberação, conceitos apresentados na subseção~\ref{subsec:filtros} e disponibilizada online junto ao toolbox \massa.\cite{MASSA}
\code{ruidosa faixa}{scripts/pecas2.2/ruidosaFaixa4.py}


\clearpage

\subsection{Bela Rugosi}\label{ap:bela}
Pequena peça musical de 96 segundos para demonstração da rugosidade causada por oscilações com frequencias $\approx 13-30 Hz$, conceitos apresentados na subseção~\ref{subsec:tvaf}. A peça é também disponibilizada online junto à toolbox \massa.\cite{MASSA}
\code{bela rugosi}{scripts/pecas2.2/bellaRugosi.py}


\clearpage

\subsection{Chorus infantil}\label{ap:chorus}
Script para demonstração do efeito \emph{chorus} (inspirado em coro de cantores) . Sintetiza 4 pequenas montagens de 8-32 segundos com os conceitos apresentados na subseção~\ref{subsec:mus2} e disponibilizada online junto ao toolbox \massa.\cite{MASSA}
\code{chorus infantil}{scripts/pecas2.2/chorusInfantil.py}


\clearpage

\subsection{ADa e SaRa}\label{ap:ada}
Pequena peça musical para demonstração da . Resulta em uma pequena peça de XX segundos. Peça demonstrativa dos conceitos apresentados na seção~\ref{varInternas} e disponibilizada online junto ao toolbox \massa.
\code{ADa e SaRa}{scripts/pecas2.2/ADa_e_SaRa.py}


%%%%%%%%%%%%%%%%%%%%%%%%%%%%%%%%%%%%%%
%%%% SECAO 2.3

\clearpage
\subsection{Intervalos entre alturas}\label{ap:intervalos}
Pequena peça musical para demonstração da . Resulta em uma pequena peça de XX segundos. Peça demonstrativa dos conceitos apresentados na seção~\ref{notasMusica} e disponibilizada online junto ao toolbox \massa.
\code{Intervalos entre alturas}{scripts/pecas2.3/intervalosEntreAlturas.py}


\clearpage
\subsection{Acorde cedo}\label{ap:acorde}
Pequena peça musical para demonstração da . Resulta em uma pequena peça de XX segundos. Peça demonstrativa dos conceitos apresentados na seção~\ref{notasMusica} e disponibilizada online junto ao toolbox \massa.
\code{Acorde cedo}{scripts/pecas2.3/acordeCedo.py}

\clearpage

\subsection{Conta ponto}\label{ap:conta}
Pequena peça musical para demonstração da . Resulta em uma pequena peça de XX segundos. Peça demonstrativa dos conceitos apresentados na seção~\ref{notasMusica} e disponibilizada online junto ao toolbox \massa.
\code{Cristais}{scripts/pecas2.3/contaPonto.py}


\clearpage

\subsection{Cristais}\label{ap:cristais}
Pequena peça musical para demonstração da . Resulta em uma pequena peça de XX segundos. Peça demonstrativa dos conceitos apresentados na seção~\ref{notasMusica} e disponibilizada online junto ao toolbox \massa.
\code{Cristais}{scripts/pecas2.3/cristais.py}


\clearpage

\subsection{Micro tom}\label{ap:micro}
Pequena peça musical para demonstração da . Resulta em uma pequena peça de XX segundos. Peça demonstrativa dos conceitos apresentados na seção~\ref{notasMusica} e disponibilizada online junto ao toolbox \massa.
\code{Cristais}{scripts/pecas2.3/microTom.py}


\clearpage

\subsection{Poli Hit Mia}\label{ap:poli}
Pequena peça musical para demonstração da . Resulta em uma pequena peça de XX segundos. Peça demonstrativa dos conceitos apresentados na seção~\ref{notasMusica} e disponibilizada online junto ao toolbox \massa.
\code{Cristais}{scripts/pecas2.3/poliHitMia.py}


\clearpage

\subsection{Dirracional}\label{ap:dirracional}
Pequena peça musical para demonstração da . Resulta em uma pequena peça de XX segundos. Peça demonstrativa dos conceitos apresentados na seção~\ref{notasMusica} e disponibilizada online junto ao toolbox \massa.
\code{Cristais}{scripts/pecas2.3/dirracional.py}


